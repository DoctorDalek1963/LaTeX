\documentclass[a4paper]{article}
\usepackage[utf8]{inputenc}
\usepackage{amsmath, amssymb}
\usepackage{cancel}
\usepackage[nodayofweek]{datetime}

% Set size of text area with total parameter
\usepackage[a4paper, total={155mm, 255mm}]{geometry}

\title{Proving The Power Rule}
\author{Dyson}
\date{\today}

\newcommand{\deriv}{\dfrac{\text{d}}{\text{d}x}}

\begin{document}

\maketitle

% Set paragraph spacing here to avoid messing with title
\setlength{\parindent}{0em}
\setlength{\parskip}{1em}

\section{The Conjecture}

The power rule states that for all $\displaystyle n \in \mathbb{Z}, \deriv x^n = nx^{n - 1}$. We want to prove this from first principles.

\section{The Proof}

\subsection{For The Naturals}

Proving the power rule for $n \in \mathbb{N}, 0 \notin \mathbb{N}$ is relatively easy and just involves some simple binomial expansion.\\

Let $f(x) = x^n$.

We know that the derivate $f'(x)$ of $f(x)$ is defined as $$f'(x) = \lim_{h \to 0} \left( \frac{f(x + h) - f(x)}{h} \right)$$

If we plug in our $f(x) = x^n$, then we get $$f'(x) = \lim_{h \to 0} \left( \frac{(x + h)^n - x^n}{h} \right)$$

We need to cancel the $h$ before we let it go to 0. We can do this by expanding the binomial $(x + h)^n$ in the numerator like so: $$(x + h)^n = x^n + nx^{n - 1}h + \binom{n}{2}x^{n - 2}h^2 + \cdots + nxh^{n - 1} + h^n$$

We now have an $x^n$ term and a $-x^n$ term in the numerator. These cancel to give us $$\lim_{h \to 0} \left( \frac{nx^{n - 1}h + \binom{n}{2}x^{n - 2}h^2 + \cdots + nxh^{n - 1} + h^n}{h} \right)$$

We can then factor out $h$ from the numerator and cancel like so: $$\lim_{h \to 0} \left( \frac{\cancel{h} \left( nx^{n - 1} + \binom{n}{2}x^{n - 2}h + \cdots + nxh^{n - 2} + h^{n - 1} \right)}{\cancel{h}} \right)$$
$$=\lim_{h \to 0} \left(nx^{n - 1} + \binom{n}{2}x^{n - 2}h + \cdots + nxh^{n - 2} + h^{n - 1} \right)$$

We can now let $h$ go to 0 and thereby show that $f'(x) = nx^{n - 1}$

\hspace*{\fill}$\square$

\subsection{For All Integers}

Proving the power rule for all $n \in \mathbb{Z}$ is a bit more complicated.

We know that the power rule would say $$\deriv x^0 = 0x^{-1} = 0$$ We also know that $x^0$ is always 1, and the derivative of a constant is always 0, so the power rule holds for $n = 0$.

To prove it for negative integers, I'm going to prove that $$\deriv x^{-n} = -nx^{-n-1}, n \in \mathbb{N}$$ because this is easier to prove and will expand the proof to all integers.\\

Let $f(x) = x^{-n}$.

We plug this $f(x)$ into the definition and get $$f'(x) = \lim_{h \to 0}\left( \frac{(x + h)^{-n} - x^{-n}}{h} \right)$$

If it's possible to expand binomials with negative powers, I don't know how to do it, but I do know that $a^{-b} = \dfrac{1}{a^b}$, so we'll use that and focus on the numerator for now.

\begin{gather*}
(x + h)^{-n} - x^{-n} = \frac{1}{(x + h)^n} - \frac{1}{x^n}\\[0.5em]
= \frac{x^n}{x^n(x + h)^n} - \frac{(x + h)^n}{x^n(x + h)^n}\\[0.5em]
= \frac{x^n - (x + h)^n}{x^n(x + h)^n}
\end{gather*}

We're going to expand and simplify the numerator, so for the sake of simplicity, I'm leaving the denominator unexpanded for now.

\begin{gather*}
\frac{x^n - (x^n + nx^{n - 1}h + \binom{n}{2}x^{n - 2}h^2 + \cdots + nxh^{n - 1} + h^n)}{x^n(x + h)^n}\\[0.5em]
= \frac{-nx^{n - 1}h - \binom{n}{2}x^{n - 2}h^2 - \cdots - nxh^{n - 1} - h^n}{x^n(x + h)^n}
\end{gather*}

Now, we're going to re-introduce $h$ before expanding the denominator.

Dividing a fraction by $h$ is the same as just multiplying the denominator by $h$.

\begin{gather*}
\frac{-nx^{n - 1}h - \binom{n}{2}x^{n - 2}h^2 - \cdots - nxh^{n - 1} - h^n}{x^n(x + h)^n} \div h\\[0.5em]
= \frac{-nx^{n - 1}h - \binom{n}{2}x^{n - 2}h^2 - \cdots - nxh^{n - 1} - h^n}{hx^n(x + h)^n}
\end{gather*}

We can factor a $h$ out from the numerator and get $$\frac{h(-nx^{n - 1} - \binom{n}{2}x^{n - 2}h - \cdots - nxh^{n - 2} - h^{n - 1})}{hx^n(x + h)^n}$$

We can now cancel the $h$ and expand the denominator.

\begin{gather*}
\frac{-nx^{n - 1} - \binom{n}{2}x^{n - 2}h - \cdots - nxh^{n - 2} - h^{n - 1}}{x^n(x^n + nx^{n - 1}h + \binom{n}{2}x^{n - 2}h^2 + \cdots + nxh^{n - 1} + h^n)}\\[0.5em]
= \frac{-nx^{n - 1} - \binom{n}{2}x^{n - 2}h - \cdots - nxh^{n - 2} - h^{n - 1}}{x^{2n} + nx^{2n - 1}h + \binom{n}{2}x^{2n - 2}h^2 + \cdots + nx^{n + 1}h^{n - 1} + x^nh^n}
\end{gather*}

Now, we let $h$ go to 0 to get rid of all the $h$ terms and get left with $$\dfrac{-nx^{n - 1}}{x^{2n}}$$

Because $\dfrac{1}{x^{2n}} = x^{-2n}$, we can rewrite this as $$-nx^{n - 1}x^{-2n} = -nx^{n - 1 - 2n} = -nx^{-n - 1}$$

\hspace*{\fill}$\square$

The proof for negative integers is a bit longer and more involved. There's probably a much more elegant proof, but I'm pretty sure this one works, and I'm happy with it.

Thus, I have proved that for all $n \in \mathbb{Z}$, $\deriv x^n = nx^{n - 1}$.

\end{document}
