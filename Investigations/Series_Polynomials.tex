\documentclass[a4paper]{article}
\usepackage[utf8]{inputenc}
\usepackage{amsmath, amssymb}
\usepackage[nodayofweek]{datetime}

\usepackage{minted}
\usepackage{hyperref}

\usepackage[backend=biber, style=numeric]{biblatex}
\addbibresource{series_polynomials.bib}

% Set size of text area with total parameter
\usepackage[a4paper, total={135mm, 255mm}]{geometry}

\title{Series Polynomials}
\author{Dyson}
\date{\today}

\newcommand{\sn}{\sum\limits_{r=1}^{n}}
\newcommand{\inn}{\in \mathbb{N}}
\newcommand{\oo}[1]{\frac{1}{#1}}

\newtheorem{lemma}{Lemma}
\newtheorem{sublemma}{Lemma}[lemma]
\newtheorem{conjecture}{Conjecture}

\begin{document}

\maketitle

% Set paragraph spacing here to avoid messing with title
\setlength{\parindent}{0em}
\setlength{\parskip}{1em}

The challenge question in Exercise 3B of the Edexcel AS and A Level Further Maths Core book is about series sums of polynomials.

Part $a$ of this question asks for polynomials $f_2(x)$, $f_3(x)$, $f_4(x)$ such that for every $n \inn$, $n > 1$,
\begin{gather*}
\sn f_2(r) = n^2\\
\sn f_3(r) = n^3\\
\sn f_4(r) = n^4
\end{gather*}

Finding these polynomials reveals a very interesting pattern.

\section{Finding Polynomials}

Throughout this paper, all $n \inn$, $n > 1$.

Let's first establish some basic lemmas about series.

\begin{lemma}
$\sn (f(r) + g(r)) = \sn f(r) + \sn g(r)$
\label{lem:add}
\end{lemma}

\begin{lemma}
$\sn kf(r) = k \sn f(r)$
\label{lem:mult}
\end{lemma}

\begin{lemma}
$\sn 1 = n$
\label{lem:sum_1}
\end{lemma}

\begin{lemma}
$\sn r = \oo{2}n(n + 1)$
\label{lem:sum_r}
\end{lemma}

\begin{lemma}
$\sn r^2 = \oo{6}n(n + 1)(2n + 1)$
\label{lem:sum_r2}
\end{lemma}

\begin{lemma}
$\sn r^3 = \oo{4}n^2(n + 1)^2$
\label{lem:sum_r3}
\end{lemma}

\subsection{Finding $f_2(r)$}
We want a polynomial $f_2(r)$ such that $\sn f_2(r) = n^2$.

We already have, by Lemma~\ref{lem:sum_r}, $$\sn r = \oo{2}n(n + 1) = \oo{2}\left(n^2 + n\right)$$

We need to get rid of the $\oo{2}$, which we can do by multiplying by 2 to get $$2 \sn r = \sn 2r = n^2 + n$$

Then, we just need to get rid of the $n$.

We know, by Lemma~\ref{lem:sum_1}, that $\sn 1 = n$, so
\begin{gather*}
\sn 2r - \sn 1 = \sn (2r - 1) = n^2\\
\therefore f_2(r) = 2r - 1
\end{gather*}

\subsection{Finding $f_3(r)$}

Next, we want a polynomial $f_3(r)$ such that $\sn f_3(r) = n^3$.

Similarly to with $f_2(r)$, by Lemma~\ref{lem:sum_r2}, we already have
\begin{gather*}
\sn r^2 = \oo{6}n(n + 1)(2n + 1)\\
= \oo{3}n^3 + \oo{2}n^2 + \oo{6}n\\[0.2em]
= \oo{3}\left(n^3 + \frac{3}{2}n^2 + \oo{2}n\right)
\end{gather*}

We can then multiply by $3$ to remove the $\oo{3}$ and get $$3 \sn r^2 = \sn 3r^2 = n^3 + \frac{3}{2}n^2 + \oo{2}n$$

To get rid of the $\oo{2}n$, we can just do $$\sn 3r^2 - \sn \oo{2} = \sn \left(3r^2 - \oo{2}\right) = n^3 + \frac{3}{2}n^2$$

We know that $\sn f_2(r) = n^2$, so we simply need to subtract $\frac{3}{2}f_2(r)$ from our polynomial to get rid of the resultant $\frac{3}{2}n^2$.
\begin{gather*}
3r^2 - \oo{2} - \frac{3}{2}(2r - 1) = 3r^2 - 3r + 1\\
\therefore f_3(r) = 3r^2 - 3r + 1
\end{gather*}

\subsection{Finding $f_4(r)$}

Next, we want a polynomial $f_4(r)$ such that $\sn f_4(r) = n^4$.

We know, by Lemma~\ref{lem:sum_r3}, that
\begin{gather*}
\sn r^3 = \oo{4}n^2(n + 1)^2\\
= \oo{4}\left(n^4 + 2n^3 + n^2\right)
\end{gather*}

We can multiply by $4$ to get $$4 \sn r^3 = \sn 4r^3 = n^4 + 2n^3 + n^2$$

We can then get $n^4$ on its own by subtracting $2f_3(r)$ and $f_2(r)$ from $4r^3$.
\begin{gather*}
4r^3 - 2\left(3r^2 - 3r + 1\right) - (2r - 1)\\
= 4r^3 - 6r^2 + 6r - 2 - 2r + 1\\
= 4r^3 - 6r^2 + 4r - 1\\
\therefore f_4(r) = 4r^3 - 6r^2 + 4r - 1
\end{gather*}

\section{Finding Patterns}

We can find $f_1(r)$, where $\sn f_1(r) = n^1 = n$ to trivially be $1$.

These are our polynomials:
\begin{alignat*}{2}
f_1(r) &= 1\\[0.5em]
f_2(r) &= 2r - 1\\[0.5em]
f_3(r) &= 3r^2 - 3r + 1\\[0.5em]
f_4(r) &= 4r^3 - 6r^2 + 4r - 1
\end{alignat*}

After looking at these for a while, we can notice a few things. Firstly, the constants are always $\pm 1$, and the signs of these constants alternate with increasing degrees of polynomial. In fact, all the signs alternate.

Secondly, we can notice that the first term of $f_a(r)$ is always of the form $ar^{a-1}$.

However, the most interesting thing to notice with these polynomials is that the coefficients look like binomial expansions of $(r-1)^a$, albeit with the leading term removed and the signs flipped.

We can continue this pattern to make a conjecture about $f_5(r)$.
\begin{conjecture}
$f_5(r) = 5r^4 - 10r^3 + 10r^2 - 5r + 1$
\label{conj:f5}
\end{conjecture}

To find a general form, we have to think about how we go from $(r-1)^a$ to these polynomials.

Lets look at the example of $f_4(r)$. $$(r-1)^4 = r^4 - 4r^3 + 6r^2 - 4r + 1$$

We have the change the signs of all of these, so we get $$-(r-1)^4 = -r^4 + 4r^3 - 6r^2 + 4r - 1$$ Then we have to remove the leading $-r^4$ and we get $-(r-1)^4 + r^4$, or more simply, $r^4 - (r-1)^4$.

This form of $r^a - (r-1)^a$ gives the results seen previously for $f_a(r)$, so we can conjecture that this pattern continues for all $a \in \mathbb{Z}^+$.

$\mathbb{Z}^+$ is simply the set of positive integers, not including 0. This is simpler than writing $a \inn, a > 0$.

Rather than conjecturing about $f_a(r)$, we want to conjecture about its sum to $n$.
\begin{conjecture}
$\displaystyle \forall a, n \in \mathbb{Z}^+, n > 1, \sn \left(r^a - (r - 1)^a\right) = n^a$
\label{conj:fa_sum}
\end{conjecture}

\subsection{Proving Conjecture~\ref{conj:f5}}

In order to prove Conjecture~\ref{conj:f5}, we need to know the formula for $\sn r^4$. Wolfram Alpha says:

\begin{lemma}
$\sn r^4 = \oo{30}n(n + 1)(2n + 1)\left(3n^2 + 3n - 1\right)$
\label{lem:sum_r4}
\end{lemma}

This can then be expanded to give
\begin{gather*}
\oo{5}n^5 + \oo{2}n^4 + \oo{3}n^3 - \oo{30}n\\[0.2em]
= \oo{5}\left(n^5 + \frac{5}{2}n^4 + \frac{5}{3}n^3 - \oo{6}n\right)
\end{gather*}

We can multiply by 5 to get $$5 \sn r^4 = \sn 5r^4 = n^5 + \frac{5}{2}n^4 + \frac{5}{3}n^3 - \oo{6}n$$

Now, we just need to get rid of the other terms to get a polynomial $f_5(r)$ such that $\sn f_5(r) = n^5$.
\begin{gather*}
f_5(r) = 5r^4 - \frac{5}{2}f_4(r) - \frac{5}{3}f_3(r) + \oo{6}f_1(r)\\
= 5r^4 - \frac{5}{2}\left(4r^3 - 6r^2 + 4r - 1\right) - \frac{5}{3}\left(3r^2 - 3r + 1\right) + \oo{6}\\
= 5r^4 - 10r^3 + 10r^2 - 5r + 1
\end{gather*}

This proves Conjecture~\ref{conj:f5}.

However, to prove Conjecture~\ref{conj:fa_sum}, I think I need to find more patterns.

\section{Finding Patterns (again)}

Let's look at sums of powers of $r$. By Lemmas 3 to 7, we have
\begin{alignat*}{3}
&\sn r^0 &&= n &&= n\\
&\sn r^1 &&= \oo{2}n(n + 1) &&= \oo{2}n^2 + \oo{2}n\\
&\sn r^2 &&= \oo{6}n(n + 1)(2n + 1) &&= \oo{3}n^3 + \oo{2}n^2 + \oo{6}n\\
&\sn r^3 &&= \oo{4}n^2(n + 1)^2 &&= \oo{4}n^4 + \oo{2}n^3 + \oo{4}n^2\\
&\sn r^4 &&= \oo{30}n(n + 1)(2n + 1)\left(3n^2 + 3n - 1\right) &&= \oo{5}n^5 + \oo{2}n^4 + \oo{3}n^3 - \oo{30}n
\end{alignat*}

I can't see much of a pattern here, but the first term of $\sn r^b$ always seems to be $\oo{b+1}n^{b+1}$ and the second term always seems to be $\oo{2}n^{b}$, except in the case of $b = 0$.

I don't think I would able to find a formula for $\sn r^b$ on my own, but in doing some research on this topic, I found this wiki page on \href{https://brilliant.org/wiki/sum-of-n-n2-or-n3/#faulhabers-formula}{brilliant.org} about sums which mentions Faulhaber's formula. I shall rewrite it here. $$\sn r^b = \oo{b+1} \sum_{j=0}^b (-1)^j \binom{b+1}{j} B_j n^{b+1-j}$$ where $B_j$ is the $j$th Bernoulli number.

In doing further research, I found that this is were the Bernoulli numbers where found by Jakob Bernoulli in \textit{Ars Conjectandi}, published in 1713. He was trying to find a general formula for $\sn r^b$ and found these numbers. He could not relate them to anything, and de Moivre named them after him.\cite{wikipedia-bernoulli-numbers}

Not only that, but Jakob Bernoulli mentioned Faulhaber by name in \textit{Ars Conjectandi}. People have wondered about this problem long before me, and this is where the famous Bernoulli numbers originally came from.

Bernoulli would have written $\sn r^b$ as $$\sn r^b = \sum_{r=0}^b \frac{B_r}{r!} b^{\underline{r-1}} n^{b-r+1}$$ where $p^{\underline{q}}$ is the falling factorial $p \times (p-1) \times (p-2) \times \cdots \times (p-q+1)$.

\section{Proving Conjecture~\ref{conj:fa_sum}}

Proving Conjecture~\ref{conj:fa_sum} will undoubtedly be hard, and will definitely require knowledge of the Bernoulli numbers. Unfortunately, there seems to be no easy formula for $B_k$. The Bernoulli numbers are usually either defined with a recurrence relation, or with a generating function, or in terms of things like the Riemann Zeta function. So, nothing friendly. Unfortunately, it seems like proving this conjecture is a bit out of my reach currently. But I posited it, and I want to prove it, even if it takes ages.

\printbibliography

\end{document}
