\documentclass[a4paper]{article}
\usepackage[utf8]{inputenc}
\usepackage{amsmath, amssymb}
\usepackage[nodayofweek]{datetime}

\usepackage{minted}

% Set size of text area with total parameter
\usepackage[a4paper, total={135mm, 255mm}]{geometry}

\title{Series Polynomials}
\author{Dyson}
\date{\today}

\newcommand{\sn}{\sum\limits_{r=1}^{n}}
\newcommand{\inn}{\in \mathbb{N}}
\newcommand{\oo}[1]{\frac{1}{#1}}

\newtheorem{lemma}{Lemma}
\newtheorem{sublemma}{Lemma}[lemma]
\newtheorem{conjecture}{Conjecture}

\begin{document}

\maketitle

% Set paragraph spacing here to avoid messing with title
\setlength{\parindent}{0em}
\setlength{\parskip}{1em}

The challenge question in Exercise 3B is about series sums of polynomials.

Part $a$ of this question asks for polynomials $f_2(x)$, $f_3(x)$, $f_4(x)$ such that for every $n \inn$,
\begin{gather*}
\sn f_2(r) = n^2\\
\sn f_3(r) = n^3\\
\sn f_4(r) = n^4
\end{gather*}

Finding these polynomials reveals a very interesting pattern.

\section{Finding Polynomials}

Throughout this paper, all $n \inn$.

Let's first establish some basic lemmas about series.

\begin{lemma}
$\sn (f(r) + g(r)) = \sn f(r) + \sn g(r)$
\end{lemma}

\begin{lemma}
$\sn kf(r) = k \sn f(r)$
\end{lemma}

\begin{lemma}
$\sn 1 = n$
\end{lemma}

\begin{lemma}
$\sn r = \oo{2}n(n + 1)$
\end{lemma}

\begin{lemma}
$\sn r^2 = \oo{6}n(n + 1)(2n + 1)$
\end{lemma}

\begin{lemma}
$\sn r^3 = \oo{4}n^2(n + 1)^2$
\end{lemma}

\subsection{Finding $f_2(r)$}
We want a polynomial $f_2(r)$ such that $\sn f_2(r) = n^2$.

We already have, by Lemma 4, $$\sn r = \oo{2}n(n + 1) = \oo{2}\left(n^2 + n\right)$$

We need to get rid of the $\oo{2}$, which we can do by multiplying by 2 to get $$2 \sn r = \sn 2r = n^2 + n$$

Then, we just need to get rid of the $n$.

We know that $\sn 1 = n$, so
\begin{gather*}
\sn 2r - \sn 1 = \sn (2r - 1) = n^2\\
\therefore f_2(r) = 2r - 1
\end{gather*}

\subsection{Finding $f_3(r)$}

Next, we want a polynomial $f_3(r)$ such that $\sn f_3(r) = n^3$.

Similarly to with $f_2(r)$, by Lemma 5, we already have
\begin{gather*}
\sn r^2 = \oo{6}n(n + 1)(2n + 1)\\
= \oo{3}n^3 + \oo{2}n^2 + \oo{6}n\\[0.2em]
= \oo{3}\left(n^3 + \frac{3}{2}n^2 + \oo{2}n\right)
\end{gather*}

We can then multiply by $3$ to remove the $\oo{3}$ and get $$3 \sn r^2 = \sn 3r^2 = n^3 + \frac{3}{2}n^2 + \oo{2}n$$

To get rid of the $\oo{2}n$, we can just do $$\sn 3r^2 - \sn \oo{2} = \sn \left(3r^2 - \oo{2}\right) = n^3 + \frac{3}{2}n^2$$

We know that $\sn f_2(r) = n^2$, so we simply need to subtract $\frac{3}{2}f_2(r)$ from our polynomial to get rid of the resultant $\frac{3}{2}n^2$.

\begin{gather*}
3r^2 - \oo{2} - \frac{3}{2}(2r - 1) = 3r^2 - 3r + 1\\
\therefore f_3(r) = 3r^2 - 3r + 1
\end{gather*}

\subsection{Finding $f_4(r)$}

Next, we want a polynomial $f_4(r)$ such that $\sn f_4(r) = n^4$.

We know, by Lemma 6, that
\begin{gather*}
\sn r^3 = \oo{4}n^2(n + 1)^2\\
= \oo{4}\left(n^4 + 2n^3 + n^2\right)
\end{gather*}

We can multiply by $4$ to get $$4 \sn r^3 = \sn 4r^3 = n^4 + 2n^3 + n^2$$

We can then get $n^4$ on its own by subtracting $2f_3(r)$ and $f_2(r)$ from $4r^3$.
\begin{gather*}
4r^3 - 2\left(3r^2 - 3r + 1\right) - (2r - 1)\\
= 4r^3 - 6r^2 + 6r - 2 - 2r + 1\\
= 4r^3 - 6r^2 + 4r - 1\\
\therefore f_4(r) = 4r^3 - 6r^2 + 4r - 1
\end{gather*}

\section{Finding Patterns}

We can find $f_1(r)$, where $\sn f_1(r) = n^1 = n$ to trivially be $1$.

These are our polynomials:\\[0.5em]
$f_1(r) = 1$\\[0.5em]
$f_2(r) = 2r - 1$\\[0.5em]
$f_3(r) = 3r^2 - 3r + 1$\\[0.5em]
$f_4(r) = 4r^3 - 6r^2 + 4r - 1$

After looking at these for a while, we can notice a few things. Firstly, the constants are always $\pm 1$, and the signs of these constants alternate with increasing degrees of polynomial. In fact, all the signs alternate.

Secondly, we can notice that the first term of $f_m(r)$ is always of the form $mr^{m-1}$.

However, the most interesting thing to notice with these polynomials is that the coefficients look like binomial expansions, albeit with the leading term removed.

\vspace{3em}

\begin{conjecture}
$f_5(r) = 5r^4 - 10r^3 + 10r^2 - 5r + 1$
\end{conjecture}

And, more generally,
\begin{conjecture}
$f_m(r) = r^m - (r - 1)^m$
\end{conjecture}

This form expands to give us the full binomial expansion with alternating signs, but we subtract this from $r^m$ to flip all the signs and remove the $r^m$ term.

Conjecture 2 states: $$\forall m, n \inn, \sn \left(r^m - (r - 1)^m\right) = n^m$$

In order to prove Conjecture 1, we need to know the formula for $\sn r^4$. Wolfram Alpha says:

\begin{lemma}
$\sn r^4 = \oo{30}n(n + 1)(2n + 1)\left(3n^2 + 3n - 1\right)$
\end{lemma}

This can also be expanded to give
\begin{gather*}
\oo{5}n^5 + \oo{2}n^4 + \oo{3}n^3 - \oo{30}n\\[0.2em]
= \oo{5}\left(n^5 + \frac{5}{2}n^4 + \frac{5}{3}n^3 - \oo{6}n\right)
\end{gather*}

We can multiply by 5 to get $$5 \sn r^4 = \sn 5r^4 = n^5 + \frac{5}{2}n^4 + \frac{5}{3}n^3 - \oo{6}n$$

Now, we just need to get rid of the other terms to get a polynomial $f_5(r)$ such that $\sn f_5(r) = n^5$.

\begin{gather*}
f_5(r) = 5r^4 - \frac{5}{2}f_4(r) - \frac{5}{3}f_3(r) + \oo{6}f_1(r)\\
= 5r^4 - \frac{5}{2}\left(4r^3 - 6r^2 + 4r - 1\right) - \frac{5}{3}\left(3r^2 - 3r + 1\right) + \oo{6}\\
= 5r^4 - 10r^3 + 10r^2 - 5r + 1
\end{gather*}

This proves Conjecture 1.

However, to prove Conjecture 2, I think I need to find more patterns.

\section{Finding Patterns (again)}

Let's look at sums of powers of $r$. By Lemmas 3 to 7, we have
\begin{alignat*}{3}
&\sn r^0 &&= n &&= n\\
&\sn r^1 &&= \oo{2}n(n + 1) &&= \oo{2}n^2 + \oo{2}n\\
&\sn r^2 &&= \oo{6}n(n + 1)(2n + 1) &&= \oo{3}n^3 + \oo{2}n^2 + \oo{6}n\\
&\sn r^3 &&= \oo{4}n^2(n + 1)^2 &&= \oo{4}n^4 + \oo{2}n^3 + \oo{4}n^2\\
&\sn r^4 &&= \oo{30}n(n + 1)(2n + 1)\left(3n^2 + 3n - 1\right) &&= \oo{5}n^5 + \oo{2}n^4 + \oo{3}n^3 - \oo{30}n
\end{alignat*}

I can't see much of a pattern here, but the first term of $\sn r^m$ always seems to be $\oo{m+1}n^{m+1}$ and the second term always seems to be $\oo{2}n^{m}$, except in the case of $m = 0$.

Finding a general formula for $\sn r^m$ seems quite hard. In fact, Wolfram Alpha evaluates it as $H^{(-m)}_n$, where $H^{(k)}_n$ is the \textit{generalised harmonic number}. These numbers are related to the harmonic series, defined as $\sum\limits_{n=1}^\infty \oo{n}$.

$H^{(k)}_n$ can be defined as $$H^{(k)}_n = \sn r^{-k}$$

This means that $$H^{(-k)}_n = \sn r^k$$

This is just rephrasing the same thing. This is useless. Thanks Wolfram Alpha.

Asking SymPy to evaluate \texttt{Sum(r**m - (r - 1)**m, (r, 1, n))} gives $-0^m + n^m$, so there's clearly an algorithm to do this, but I don't know how to prove this result. I also don't know why it returns $-0^m$ as part of the answer.

I believe that proving Conjecture 2 is possible, but I don't think I currently have the tools to do so.

\section{Part \textit{b}}

This question does have a part $b$, which I should probably address quickly. It says:

Hence, show that for any linear, quadratic, or cubic polynomial $h(x)$, there exists a polynomial $g(x)$ such that $\sn g(r) = n(h(n))$.

If we let $$h(n) = an^3 + bn^2 + cn + d$$ then $$n(h(n)) = an^4 + bn^3 + cn^2 + dn$$

Then, $g(r)$ is simply $$af_4(r) + bf_3(r) + cf_2(r) + df_1(r)$$

This idea extends to linear and quadratic polynomials $h(x)$ and if we assume Conjecture 2 to be true, meaning we can find any $f_m(r)$, then this idea extends to any degree polynomial.

\end{document}
