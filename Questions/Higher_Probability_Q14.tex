\documentclass[a4paper]{article}
\usepackage[utf8]{inputenc}
\usepackage{amsmath}
\usepackage[nodayofweek]{datetime}

\usepackage{minted}

% Set size of text area with total parameter
\usepackage[a4paper, total={155mm, 255mm}]{geometry}

\title{Higher Probability Q14}
\author{Dyson}
\date{\today}

\newcommand{\oo}[1]{\frac{1}{#1}}

\begin{document}

\maketitle

% Set paragraph spacing here to avoid messing with title
\setlength{\parindent}{0em}
\setlength{\parskip}{1em}

\section{The Question}

Higher Probability Question 14 in the A Level Summer Work is quite an interesting question. It says:

David has designed a game.\\
He uses a fair 6-sided dice and a fair 5-sided spinner.\\
The dice is numbered 1 to 6.\\
The spinner is numbered 1 to 5.

Each player rolls the dice once and spins the spinner once.\\
A player can win £5 or £2.

For a player to win £5, they must roll a 5 \textbf{and} spin a 5.\\
For a player to win £2, they must roll a 1 \textbf{or} spin a 1 \textbf{or} both.

David expects 30 people will play his game.\\
Each person will pay David £1 to play the game.

Work out how much profit David can expect to make.

\hspace*{\fill}(4 marks)

\section{The Original Working}

Initially, my thinking was: each of the 30 people pays £1 to play, so we have a £30 gross profit. Then, the probability of rolling a 5 is $\oo{6}$, the probability of spinning a 5 is $\oo{5}$, so the probability of both is $\oo{6} \times \oo{5} = \oo{30}$. This means we can expect 1 of the 30 people to win £5, so our net profit at this point becomes £25.

The probability of rolling a 1 is $\oo{6}$, the probability of spinning a 1 is $\oo{5}$, and the probability of both is therefore $\oo{6} \times \oo{5} = \oo{30}$. Then, to get the probability of A \textbf{or} B \textbf{or} C, I added these probabilities together to get $\frac{2}{5}$. This is $\frac{12}{30}$, meaning 12 people win £2. This means we subtract £24 from our net profit and get a final profit of £1.

Now, this seemed off to me. It just didn't feel right that David would only make £1 profit. So, I wrote some Julia code to simulate it.

\newpage

\section{The Improved Working}

\begin{minted}{julia}
using Statistics

function game()
	total = 0

	for _ in 1:30
		# Add one to the total for each person
		total += 1

		# Roll a die and spin a spinner
		d = rand(1:6)
		s = rand(1:5)

		# Check the conditions and subtract the relevant amounts from the total
		if d == 5 && s == 5
			total -= 5
		elseif d == 1 || s == 1 || (d == 1 && s == 1)
			total -= 2
		end
	end

	return total
end

println(mean([game() for _ in 1:10000]))
\end{minted}

This code consistently gave me a mean of £5. I spent about an hour carefully checking through my code and making sure that everything was correct and eventually concluded that £5 was the real answer and my maths must have been wrong.

After some thinking, I realised that the \textit{both} part of the second condition is irrelevant. If a person both rolls a 1 \textbf{and} spins a 1, then they must have already rolled a 1 \textbf{or} spun a 1. We can show this with some boolean algebra.

Let $R$ be whether the person has rolled a 1 and let $S$ be whether the person has spun a 1. Then, our expression is $R \vee S \vee (R \wedge S)$. By associativity of $\vee$, we can write this as $R \vee (S \vee (R \wedge S))$. Then, by absorption, $S \vee (R \wedge S) = S$, so our expression gets simplified to $R \vee S$. Therefore, the \textit{both} part of the second condition can be ignored.

Changing the code confirms that by removing the check for both, the answer doesn't change. However, this probability of both \textit{must} be removed from the probability calculation. This gives us $P(\text{£}2) = \frac{11}{30}$, so 11 people win £2 and we get a final total of £3. This is still wrong.

I had to do some research on combining probabilities and learned that $P(A \vee B)$ only equals $P(A) + P(B)$ if $A$ and $B$ are dependent events, meaning they influence each other. If $A$ happens, then $B$ cannot. This is not the case with this question and we cannot simply add the probabilities of independent events like this.

In the platonic world of perfect probabilities with 30 people playing the game, 5 people roll a 1 and 6 people spin a 1. This does \textbf{not} mean that 11 people rolled a 1 \textbf{or} spun a 1. This is because $P(R \wedge S) = P(R) \times P(S) = \oo{6} \times \oo{5} = \oo{30}$. That means that 1 person rolled a 1 \textbf{and} spun a 1, so this person gets double counted when we simply add the probabilities.

\newpage

This means that for two independent events $A$ and $B$, $P(A \vee B) = P(A) + P(B) - P(A \wedge B)$. Thus, $$P(R \vee S \vee (R \wedge S)) = P(R \vee S)$$
$$= P(R) + P(S) - P(R \wedge S)$$
$$= \oo{6} + \oo{5} - \oo{30} = \oo{3}$$

This means that $\oo{3}$ of the 30 people win £2, meaning we only take £20 from our £25 net profit, finally getting the £5 answer.

Quite frankly, this question was a bitch.

\end{document}
