% This file is called psm.tex because most of this advice comes from a course I'm on, called Problem Solving Matters

\documentclass[../main.tex]{subfile}

\begin{document}

\topictitle{Problem Solving Matters}

\begin{figure}[h]
\centering
\begin{minipage}{0.85\linewidth}
\sectitle{General Tips}

\begin{itemize}
	\item Be lazy; only do necessary work
	\item Write in sentences to explain (especially in proofs)
	\item Avoid long and/or complicated calculations
	\item Draw diagrams and make them big
	\item In diagrams, label things and add lines
	\item Look for similar shapes (often triangles)
\end{itemize}

\sectitle{Tips For Sketching Graphs}

\begin{itemize}
	\item Look for symmetries
	\item Think about periodicity
	\item Look for turning points (0 derivative)
	\item Look for asymptotes
	\item Try values of $x$ like 0, 1, -1, etc.
	\item If there's a trig function involved, try multiples of $\pi$
	\item See what happens when $x$ tends to 0 or $\pm \infty$
\end{itemize}

\sectitle{Things To Remember}

\begin{itemize}
	\item $\log_a b \times \log_b a = 1$
	\item $\log_{a^c} b^c = \log_a b$
	\item When graphing $y^2 = f(x)$, draw the positive branch of $y = \sqrt{f(x)}$ \textit{and} reflect it in the $x$ axis
	\item $\log x$ is negative when $0 < x < 1$
\end{itemize}
\end{minipage}
\end{figure}

\end{document}
