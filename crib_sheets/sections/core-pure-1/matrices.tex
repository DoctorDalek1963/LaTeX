\documentclass[../main.tex]{subfile}

\begin{document}

\tikzset{
	generic matrix style/.style={
		matrix of math nodes,
		row sep=1mm,
		nodes={minimum width=1.4em, minimum height=1.8ex}
	},
	matrix style/.style={
		generic matrix style,
		left delimiter={(}, right delimiter={)},
	},
	determinant style/.style={
		generic matrix style,
		left delimiter={|}, right delimiter={|},
	}
}

\topictitle{Matrices}

\sectitle{Multiplication}

The size of a matrix is described as $h \times w$, so $\begin{pmatrix}1 & 2 & 3\\ 4 & 5 & 6\end{pmatrix}$ is $2 \times 3$ but $\begin{pmatrix}1 & 2\\ 3 & 4\\ 5 & 6\end{pmatrix}$ is $3 \times 2$. To multiply a $h_1 \times w_1$ matrix by a $h_2 \times w_2$ matrix, we require that $w_1 = h_2$; the product matrix will then be of size $h_1 \times w_2$.

Addition of matrices is done element-wise, and requires both matrices to be of the same size.

Multiplication is more complicated. Use the rows of the matrix on the left and the columns of the matrix on the right, take their dot product, and put the answer in the corresponding slot of the answer matrix.

\vspace{-5ex}
\begin{center}
	\tikzset{
		highlight/.style={
			rectangle, rounded corners,
			color=blue!40, draw,
			fill=blue!25, fill opacity=0.4,
			thick, inner sep=0pt
		},
	}

	\begin{equation*}
	\renewcommand{\arraystretch}{1.5}
	\begin{tikzpicture}[baseline=(m1.center)]
		\hspace{1.8em}
		\matrix (m1) [matrix style] {
			a & b\\
			c & d\\
		};
		\node [highlight, fit=(m1-1-1.north west) (m1-1-2.south east)] {};

		\matrix (m2) [matrix style, right=of m1] {
			e & f\\
			g & h\\
		};
		\node [highlight, fit=(m2-1-1.north west) (m2-2-1.south east)] {};

		\hspace{-1.8em}
		\node [right=of m2] (equals) {=};
		\hspace{-1.8em}

		\matrix (m3) [matrix style, right=of equals] {
			ae + bg & \phantom{af + bh}\\
			\phantom{ce + dg} & \phantom{cf + dh}\\
		};
		\node [highlight, fit=(m3-1-1.north west) (m3-1-1.south east)] {};
	\end{tikzpicture}
	\end{equation*}

	\begin{equation*}
	\begin{tikzpicture}[baseline=(m1.center)]
		\hspace{1.8em}
		\matrix (m1) [matrix style] {
			a & b\\
			c & d\\
		};
		\node [highlight, fit=(m1-1-1.north west) (m1-1-2.south east)] {};

		\matrix (m2) [matrix style, right=of m1] {
			e & f\\
			g & h\\
		};
		\node [highlight, fit=(m2-1-2.north west) (m2-2-2.south east)] {};

		\hspace{-1.8em}
		\node [right=of m2] (equals) {=};
		\hspace{-1.8em}

		\matrix (m3) [matrix style, right=of equals] {
			ae + bg & af + bh\\
			\phantom{ce + dg} & \phantom{cf + dh}\\
		};
		\node [highlight, fit=(m3-1-2.north west) (m3-1-2.south east)] {};
	\end{tikzpicture}
	\end{equation*}

	\begin{equation*}
	\begin{tikzpicture}[baseline=(m1.center)]
		\hspace{1.8em}
		\matrix (m1) [matrix style] {
			a & b\\
			c & d\\
		};
		\node [highlight, fit=(m1-2-1.north west) (m1-2-2.south east)] {};

		\matrix (m2) [matrix style, right=of m1] {
			e & f\\
			g & h\\
		};
		\node [highlight, fit=(m2-1-1.north west) (m2-2-1.south east)] {};

		\hspace{-1.8em}
		\node [right=of m2] (equals) {=};
		\hspace{-1.8em}

		\matrix (m3) [matrix style, right=of equals] {
			ae + bg & af + bh\\
			ce + dg & \phantom{cf + dh}\\
		};
		\node [highlight, fit=(m3-2-1.north west) (m3-2-1.south east)] {};
	\end{tikzpicture}
	\end{equation*}

	\begin{equation*}
	\begin{tikzpicture}[baseline=(m1.center)]
		\hspace{1.8em}
		\matrix (m1) [matrix style] {
			a & b\\
			c & d\\
		};
		\node [highlight, fit=(m1-2-1.north west) (m1-2-2.south east)] {};

		\matrix (m2) [matrix style, right=of m1] {
			e & f\\
			g & h\\
		};
		\node [highlight, fit=(m2-1-2.north west) (m2-2-2.south east)] {};

		\hspace{-1.8em}
		\node [right=of m2] (equals) {=};
		\hspace{-1.8em}

		\matrix (m3) [matrix style, right=of equals] {
			ae + bg & af + bh\\
			ce + dg & cf + dh\\
		};
		\node [highlight, fit=(m3-2-2.north west) (m3-2-2.south east)] {};
	\end{tikzpicture}
	\end{equation*}
\end{center}

\sectitle{Determinants}

The determinant of a $2 \times 2$ matrix can be found by taking the product of the leading diagonal and subtracting the product of the other diagonal. To remember which way round it is, I remember that the identity matrix $\begin{pmatrix}1 & 0\\ 0 & 1\end{pmatrix}$ has determinant 1.

\begin{center}
\begin{tikzpicture}[
	baseline=(m.center),
	det highlight/.style={
		rectangle, rounded corners,
		draw, fill opacity=0.4,
		thick, inner sep=-3pt
	},
	det pos highlight/.style={det highlight, color=blue!40, fill=blue!25},
	det neg highlight/.style={det highlight, color=red!40, fill=red!25},
]
	\hspace*{4.3em}

	\matrix (m) [determinant style] {
		a & b\\
		c & d\\
	};
	\node [det neg highlight, rotate fit=45, fit=(m-2-1.south west) (m-2-1.north west) (m-2-1.south east) (m-1-2.north east)] {};
	\node [det pos highlight, rotate fit=135, fit=(m-1-1.north west) (m-1-1.north east) (m-1-1.south west) (m-2-2.south east)] {};

	\hspace{-1.8em}
	\node [right=of m] (equals) {=};
	\hspace{-1.8em}

	\node [right=of equals] (pos text) {$ad$};
	\node [det pos highlight, inner sep=0pt, fit=(pos text.north west) (pos text.south east)] {};

	\hspace{-2.5em}
	\node [right=of pos text] (minus) {-};
	\hspace{-2.5em}

	\node [right=of minus] (neg text) {$bc$};
	\node [det neg highlight, inner sep=0pt, fit=(neg text.north west) (neg text.south east)] {};
\end{tikzpicture}
\end{center}

\newpage

The determinant of a $3 \times 3$ matrix is a little more complicated, but still quite simple. Take the top row, and multiply each element by the determinant of its minor matrix, flipping the signs for each term.

\vspace{-5ex}
\begin{center}
	\tikzset{
		highlight/.style={
			rectangle, rounded corners,
			color=#1!40, draw,
			fill=#1!25, fill opacity=0.4,
			thick, inner sep=0pt
		},
	}

	\begin{equation*}
	\renewcommand{\arraystretch}{1.5}
	\begin{tikzpicture}[baseline=(m.center)]
		\hspace{6.8em}
		\matrix (m) [determinant style] {
			a & b & c\\
			d & e & f\\
			g & h & i\\
		};
		\node [highlight=blue, fit=(m-1-1.north west) (m-1-1.south east)] {};
		\node [highlight=blue, fit=(m-2-2.north west) (m-3-3.south east)] {};

		\hspace{-1.8em}
		\node [right=of m] (equals) {=};
		\hspace{-2.5em}

		\node [right=of equals] (c1) {a};
		\node [highlight=blue, fit=(c1.north west) (c1.south east)] {};
		\matrix (m1) [right=of c1, xshift=-1.8em, determinant style] {
			e & f\\
			h & i\\
		};
		\node [highlight=blue, fit=(m1-1-1.north west) (m1-2-2.south east)] {};

		\begin{scope}[
			every node/.style={opacity=0.15}
		]
			\hspace{-1.8em}
			\node [right=of m1] (minus) {-};
			\hspace{-2.5em}

			\node [right=of minus] (c2) {b};
			\matrix (m2) [right=of c2, xshift=-1.8em, determinant style] {
				d & f\\
				g & i\\
			};

			\hspace{-1.8em}
			\node [right=of m2] (plus) {+};
			\hspace{-2.5em}

			\node [right=of plus] (c3) {c};
			\matrix (m3) [right=of c3, xshift=-1.8em, determinant style] {
				d & e\\
				g & h\\
			};
		\end{scope}
	\end{tikzpicture}
	\end{equation*}

	\begin{equation*}
	\renewcommand{\arraystretch}{1.5}
	\begin{tikzpicture}[baseline=(m.center)]
		\hspace{6.8em}
		\matrix (m) [determinant style] {
			a & b & c\\
			d & e & f\\
			g & h & i\\
		};
		\node [highlight=red, fit=(m-1-2.north west) (m-1-2.south east)] {};
		\node [highlight=red, fit=(m-2-1.north west) (m-3-1.south east)] {};
		\node [highlight=red, fit=(m-2-3.north west) (m-3-3.south east)] {};

		\hspace{-1.8em}
		\node [right=of m] (equals) {=};
		\hspace{-2.5em}

		\node [right=of equals] (c1) {a};
		\matrix (m1) [right=of c1, xshift=-1.8em, determinant style] {
			e & f\\
			h & i\\
		};

		\hspace{-1.8em}
		\node [right=of m1] (minus) {-};
		\hspace{-2.5em}

		\node [right=of minus] (c2) {b};
		\node [highlight=red, fit=(c2.north west) (c2.south east)] {};
		\matrix (m2) [right=of c2, xshift=-1.8em, determinant style] {
			d & f\\
			g & i\\
		};
		\node [highlight=red, fit=(m2-1-1.north west) (m2-2-2.south east)] {};

		\begin{scope}[
			every node/.style={opacity=0.15}
		]
			\hspace{-1.8em}
			\node [right=of m2] (plus) {+};
			\hspace{-2.5em}

			\node [right=of plus] (c3) {c};
			\matrix (m3) [right=of c3, xshift=-1.8em, determinant style] {
				d & e\\
				g & h\\
			};
		\end{scope}
	\end{tikzpicture}
	\end{equation*}

	\begin{equation*}
	\renewcommand{\arraystretch}{1.5}
	\begin{tikzpicture}[baseline=(m.center)]
		\hspace{6.8em}
		\matrix (m) [determinant style] {
			a & b & c\\
			d & e & f\\
			g & h & i\\
		};
		\node [highlight=blue, fit=(m-1-3.north west) (m-1-3.south east)] {};
		\node [highlight=blue, fit=(m-2-1.north west) (m-3-2.south east)] {};

		\hspace{-1.8em}
		\node [right=of m] (equals) {=};
		\hspace{-2.5em}

		\node [right=of equals] (c1) {a};
		\matrix (m1) [right=of c1, xshift=-1.8em, determinant style] {
			e & f\\
			h & i\\
		};

		\hspace{-1.8em}
		\node [right=of m1] (minus) {-};
		\hspace{-2.5em}

		\node [right=of minus] (c2) {b};
		\matrix (m2) [right=of c2, xshift=-1.8em, determinant style] {
			d & f\\
			g & i\\
		};

		\hspace{-1.8em}
		\node [right=of m2] (plus) {+};
		\hspace{-2.5em}

		\node [right=of plus] (c3) {c};
		\node [highlight=blue, fit=(c3.north west) (c3.south east)] {};
		\matrix (m3) [right=of c3, xshift=-1.8em, determinant style] {
			d & e\\
			g & h\\
		};
		\node [highlight=blue, fit=(m3-1-1.north west) (m3-2-2.south east)] {};
	\end{tikzpicture}
	\end{equation*}
\end{center}

\sectitle{Standard results}

\vspace*{-2ex}
\begin{center}
\begin{tabular}{c c}
	Reflection in $y = x$ & $\begin{pmatrix}0 & 1\\ 1 & 0\end{pmatrix}$\\
	\hphantom{} & \hphantom{}\\
	Reflection in $y = -x$ & $\begin{pmatrix}0 & -1\\ -1 & 0\end{pmatrix}$\\
	\hphantom{} & \hphantom{}\\
	2D rotation & $\begin{pmatrix}\cos\theta & -\sin\theta\\ \sin\theta & \cos\theta\end{pmatrix}$\\
	\hphantom{} & \hphantom{}\\
	3D rotation around $x$-axis & $\begin{pmatrix}1 & 0 & 0\\ 0 & \cos\theta & -\sin\theta\\ 0 & \sin\theta & \cos\theta\end{pmatrix}$\\
	\hphantom{} & \hphantom{}\\
	3D rotation around $y$-axis & $\begin{pmatrix}\cos\theta & 0 & \sin\theta\\ 0 & 1 & 0\\ -\sin\theta & 0 & \cos\theta\end{pmatrix}$\\
	\hphantom{} & \hphantom{}\\
	3D rotation around $z$-axis & $\begin{pmatrix}\cos\theta & -\sin\theta & 0\\ \sin\theta & \cos\theta & 0\\ 0 & 0 & 1\end{pmatrix}$\\
\end{tabular}
\end{center}

\newpage

\sectitle{Inverses}

The inverse of a $2 \times 2$ matrix is quite easy to find. Remember that $\mathbf{I}^{-1} = \mathbf{I}$, not $-\mathbf{I}$, so flip the leading diagonal and negate the opposite diagonal.
$$\boxed{\begin{pmatrix}a & b\\ c & d\end{pmatrix}^{-1} = \frac{1}{\det}\begin{pmatrix}d & -b\\ -c & a\end{pmatrix}}$$

The inverse of a $3 \times 3$ matrix is significantly more complicated. For a $3 \times 3$ matrix $\mathbf{A}$, $\boxed{\mathbf{A}^{-1} = \dfrac{1}{\det\mathbf{A}} \mathbf{C}^\text{T}}$, where $\mathbf{C}$ is the matrix of cofactors, formed by the matrix of minors $\mathbf{M}$.

Here's an example:

\steptitle{1}

The matrix $\displaystyle \mathbf{A} = \begin{pmatrix}1 & 3 & 1\\ 0 & 4 & 1\\ 2 & -1 & 0\end{pmatrix}$. Begin by finding its determinant, as explained previously.

$\det\mathbf{A} = 1 \begin{vmatrix}4 & 1\\ -1 & 0\end{vmatrix} - 3 \begin{vmatrix}0 & 1\\ 2 & 0\end{vmatrix} + 1 \begin{vmatrix}0 & 4\\ 2 & -1\end{vmatrix} = 1 + 6 - 8 = -1$.

\steptitle{2}

Keep the determinant handy; we'll need it later. Now we want to find the matrix of minors, $\mathbf{M}$. For each element in the original matrix $\mathbf{A}$, its equivalent element in the matrix $\mathbf{M}$ is found by crossing out the row and column containing that element in $\mathbf{A}$ and taking the determinant of the remaining elements. Like what we did to find the determinant, but without the sign flipping and multiplying by the coefficients. Therefore,
$$\mathbf{M} = \begin{pmatrix}
	\begin{vmatrix}4 & 1\\ -1 & 0\end{vmatrix} & \begin{vmatrix}0 & 1\\ 2 & 0\end{vmatrix} & \begin{vmatrix}0 & 4\\ 2 & -1\end{vmatrix}\\
	\hphantom{} & \hphantom{} & \hphantom{}\\
	\begin{vmatrix}3 & 1\\ -1 & 0\end{vmatrix} & \begin{vmatrix}1 & 1\\ 2 & 0\end{vmatrix} & \begin{vmatrix}1 & 3\\ 2 & -1\end{vmatrix}\\
	\hphantom{} & \hphantom{} & \hphantom{}\\
	\begin{vmatrix}3 & 1\\ 4 & 1\end{vmatrix} & \begin{vmatrix}1 & 1\\ 0 & 1\end{vmatrix} & \begin{vmatrix}1 & 3\\ 0 & 4\end{vmatrix}\\
\end{pmatrix}
= \begin{pmatrix}1 & -2 & -8\\ 1 & -2 & -7\\ -1 & 1 & 4\end{pmatrix}$$

\steptitle{3}

Now, we find the matrix of cofactors, $\mathbf{C}$, by flipping the signs of certain elements in the matrix of minors, $\mathbf{M}$.

Use the pattern $\begin{pmatrix}
	\textcolor{blue}{+} & \textcolor{red}{-} & \textcolor{blue}{+}\\
	\textcolor{red}{-} & \textcolor{blue}{+} & \textcolor{red}{-}\\
	\textcolor{blue}{+} & \textcolor{red}{-} & \textcolor{blue}{+}
\end{pmatrix}$ to find that $\mathbf{C} = \begin{pmatrix}1 & 2 & -8\\ -1 & -2 & 7\\ -1 & -1 & 4\end{pmatrix}$.

\steptitle{4}

Take the transpose of $\mathbf{C}$ to get $\mathbf{C}^\text{T} = \begin{pmatrix}1 & -1 & -1\\ 2 & -2 & 1\\ -8 & 7 & 4\end{pmatrix}$.

\steptitle{5}

Use the formula to find $\mathbf{A}^{-1} = \dfrac{1}{\det\mathbf{A}} \mathbf{C}^\text{T} = \begin{pmatrix}-1 & 1 & 1\\ -2 & 2 & 1\\ 8 & -7 & -4\end{pmatrix}$.

\end{document}
