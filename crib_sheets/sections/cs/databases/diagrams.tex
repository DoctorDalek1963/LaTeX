\documentclass[../main.tex]{subfile}

\begin{document}

\topictitle{Diagrams}

Boxes contain entities, or the fields of a record in that table. Relationship lines have a single line connection to the box for a \textit{one} connection, and a crow's foot connection for \textit{many}.

In this example, there is an online revision service which provides textbooks. A customer can subscribe to different tiers. A customer has personal details; a product is a book with a name, subject, level, and price; and a subscription links these two together with foreign keys for each, as well as a start and end date.

\begin{center}
\begin{tikzpicture}[
	every node/.style={font=\ttfamily},
	node distance=1.25in
]
	\matrix [
		entity=tblCustomer,
		entity anchor=tblCustomer-custID,
	] {
		\pk{custID},
		\properties{
			firstName,
			lastName,
			emailAddress,
			address,
		}
	};

	\matrix [
		entity=tblSubscription,
		entity anchor=tblSubscription-subID,
		right=of tblCustomer,
		yshift=3em,
	] {
		\pk{subID},
		\fk{custID},
		\fk{prodID},
		\properties{
			startDate,
			endDate,
		}
	};

	\matrix [
		entity=tblProduct,
		entity anchor=tblProduct-prodID,
		right=of tblSubscription,
		yshift=3em,
	] {
		\pk{prodID},
		\properties{
			prodName,
			subject,
			level,
			price,
		}
	};

	\draw [one to many] (tblCustomer-custID) +(18.5ex, 0) -- +(3.5, 0) |- (tblSubscription-custID);
	\draw [one to many] (tblProduct-prodID) +(-10ex, 0) -- +(-2.3, 0) |- ($(tblSubscription-prodID) +(18.5ex, 0)$);
\end{tikzpicture}
\end{center}

The \texttt{tblCustomer} table has personal details and a unique ID, which is its primary key. The \texttt{tblProduct} table has details about the product and its own unique ID primary key. The \texttt{tblSubscription} table has a unique ID primary key, a start and end date, and two foreign keys. Each subscription references the customer that subscribes to it and the product they subscribe to, in the form of two primary keys.

\end{document}
