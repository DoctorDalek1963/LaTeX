\documentclass[../main.tex]{subfile}

\begin{document}

\topictitle{Transaction processing}

\sectitle{Capturing data}

There are various ways of collecting data, including smart card readers, barcode readers, scanners, optical character recognition, optical mark recognition (like UKMT answer sheets), magnetic ink character recognition (like on banking cheques), and sensors.

Once data has been collected, it needs to be transferred to a database. This is often done using \textbf{Electronic Data Interchange (EDI)}, which is used to transfer data between computer systems.

\sectitle{EDI}

EDI is the computer-to-computer exchange of documents such as purchase orders, invoices and shipping documents between companies or business partners. It replaces post, email, or fax. All documents must be in a standard format so that the computers can understand them. EDI translation software may be used to translate the EDI format so the data can be input directly to a company database.

\sectitle{Transactions}

In the context of databases, a single logical operation is defined as a transaction. It may consist of several operations; for example, a customer order may consist of several order lines:
\begin{itemize}
	\item All of the order lines must be processed
	\item The quantity of each product must be adjusted on the stock file
	\item Credit card details must be checked
	\item Payments must be accepted or rejected
\end{itemize}

What happens if the stock file has been updated and the system crashes before payment is processed?

\sectitle{ACID}

\textbf{A}tomicity, \textbf{C}onsistency, \textbf{I}solation, \textbf{D}urability: this is a set of properties to ensure that the integrity of the database is maintained under all circumstances. It guarantees that transactions are processed reliably.

\textbf{Atomicity} requires that a transaction is processed entirely or not at all. In any situation, including power cuts or hard disk crashed, it is not possible to process only part of a transaction.

\textbf{Consistency} ensures that no transaction can violate any of the defined validation rules. Referential integrity will always be upheld.

\textbf{Isolation} ensures that concurrent execution of transactions leads to the same result as if transactions were processed one after the other. This is crucial in a multi-user database.

\textbf{Durability} ensures that once a transaction has been committed, it will remain so, even in the event of a power cut. As each part of a transaction is completed, it is held in a buffer on disk until all elements of the transaction are completed. Only then will the changes to the tables actually be made.

\sectitle{Record locking and multi-user databases}

Allowing multiple suers to simultaneously access a databases could potentially cause one of the updates to be lost.

For example, when an item is to be updated, the entire block in which the record is located is read into the user's local memory, then when the record is saved, the block is rewritten to the file server. This can easily result in overwriting.

Record locking prevents simultaneous access to objects in a database in order to prevent updates being lost or inconsistencies in the data arising. A record is locked when a user retrieves it for editing or updating. Anyone else attempting to retrieve it is denied access until the transaction is completed or cancelled. A bit like a mutex.

If Alice is trying to make a transfer from A to B and Bob is trying to make a transfer from B to A, then we get a deadlock. Neither can write their transaction because the target record is held by another user. This is a deadlock.

\end{document}
