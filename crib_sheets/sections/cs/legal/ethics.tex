\documentclass[../main.tex]{subfile}

\begin{document}

\topictitle{Ethical, moral, and cultural issues}

\sectitle{Commerce}

Photography was once a big employer because a lot of people were needed to process photographs. Kodak employed 145,000 people in 1989, but only 8000 in 2015. In 2013, Instagram had only 13 full time staff and was sold to Facebook for \$1bn.

Music, video, and publishing is open to anyone who uses smart technology. Consumers pay less, or nothing, therefore the artists make less, or nothing.

$\nicefrac{3}{4}$ of British consumers purchase goods online. The UK spends the most money per capita per year than any other country. Even higher than the US.

You no longer need inside knowledge to find the best deal, just a price comparison site. There are comparison sites to compare comparison sites. Economists describe \textbf{competition} as working best when buyers and sellers all have \textbf{perfect information} about price, usefulness, quality, and production methods. This is obviously much easier with the advent of the internet.

Questions about production methods should include: \begin{itemize}
	\item Country of manufacture
	\item Use of child labour
	\item Use of animal testing
	\item Use of recycled or organic ingredients
	\item Renewable energy use
	\item Charitable or community activity of producers
\end{itemize}

\sectitle{Social media}

In 2015, Facebook had a total revenue of \$18bn, but it's free to use, so most of that money came from advertising and selling personal data. Advertisers pay to target particular types of users.

Facebook's assets are its huge userbase and the data it stores about each individual user - their likes, locations, age, and friends. A famous saying in advertising is \enquote{Half the money I spend on advertising is waster; the trouble is I don't know which half.} But with data mining and digital tracking, the platform knows who clicked which advert.

Estonia has developed a sophisticated system of e-Government, from national to local levels. 95\% of Estonian tax declarations are filed electronically. In the 2015 Parliamentary Elections, internet voting accounted for 30.5\% of the votes cast. Estonians worldwide cast their votes from 116 different countries. A nationwide e-Health system integrates data to create a common record for each patient.
\begin{itemize}
	\item How are users authenticated?
	\item How is data kept safe from hackers, including those from enemy nations?
	\item How reliable is the technology?
	\item Will citizens trust the authorities and their technology?
	\item Will costs be matched by savings?
\end{itemize}

\sectitle{Robotics}

Solving the technological problems of robotics can bring a focus onto ethical questions. Ethics is concerned with what is good for individuals and society and is also described as \textit{moral philosophy}. An example is how we program autonomous robots: driver-less cars, drones, robotic surgeons, and security systems all raise questions.

Isaac Asimov described three laws for robots\footnote{Asimov was an author, and his laws were designed to be used in his books and shown to be ineffective as a service to the narrative. These laws will designed to show that no reasonable set of moral laws can be fully consist and appropriate for the real world.}:
\begin{enumerate}
	\item A robot may not injure a human being, or, through inaction, allow a human being to come to harm
	\item A robot must obey the orders given to it by human beings except where such orders would conflict with the first law
	\item A robot must protect its own existence as long as such protection does not conflict with the first or second laws
\end{enumerate}

\sectitle{Ethical frameworks and AI}

If a manufacturer offers different versions of its moral algorithm, and a buyer knowingly chooses one of them, then who is to blame when the algorithm makes a harmful decision?

AI algorithms can analyse social media, CVs, credit ratings, buying history, postcode data, and more. These processes were previously done by hand. Employers, universities, law enforcement, and insurance companies all use algorithms and data to some extent.

When can it be said that a computer is intelligent? What does it mean for a computer to be intelligent? Perhaps we could use the Turing Test. But not all humans behave intelligently, and not all intelligent behaviour is performed by humans.

\sectitle{Environmental effects}

Digital devices use up vast quantities of precious metals and other resources. Data centres around the world (\enquote{the cloud}) use more energy than the whole of the UK uses for heat, light, transport, etc. Data servers have a bigger carbon footprint than the global aviation industry. Do the positive effects of modern technology outweigh the negative environmental effects?

Digital control systems allow control of energy use in the home, industry, and transport.

In 2014, 14\% of 30 million working adults (4.2 million people) worked from home. About half were managers or professionals. Fewer commuters means less energy used for travel. Heating individual homes may mean more energy being used than heating a shared office. Less travel may mean less stress. Lonelier people and indirect communication may mean more stress.

\end{document}
