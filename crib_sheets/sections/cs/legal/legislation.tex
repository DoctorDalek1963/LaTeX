\documentclass[../main.tex]{subfile}

\begin{document}

\topictitle{Legislation}

Laws can be national or international. A lot of EU law has been adopted by the UK and carried through even after Brexit.

A civil case may end up with one side being awarded damages. A criminal case may end up with a fine or a prison sentence.

It is illegal to: \begin{itemize}
	\item Store or process personal data without keeping it secure, among other conditions
	\item Make or trade in hack tools - hardware or software
	\item Make digital copies of other people's work without permission
	\item Intercept messages such as phone calls or emails without legal authority to do so
\end{itemize}

\sectitle{The Big Four Acts}

The \hyperref[data-protection-act]{\textbf{Data Protection Act 1998}} controls the way that data about living people is stored and processed.

The \hyperref[computer-misuse-act]{\textbf{Computer Misuse Act 1990}} makes it an offence to access or modify computer material without permission.

The \hyperref[copyright-designs-and-patents-act]{\textbf{Copyright, Designs and Patents Act 1988}} covers the copying and use of other people's work.

The \hyperref[regulation-of-investigatory-powers-act]{\textbf{Regulation of Investigatory Powers Act 2000}} regulates surveillance and investigation, and covers the interception of communications.

\sectitle{The Data Protection Act 1998\label{data-protection-act}}

This act controls the way that data about living people is stored and processed.

Storage and processing of personal details must: \begin{enumerate}
	\item Be fair and lawful
	\item Relevant and not excessive
	\item Accurate and up to date
	\item Only kept as long as needed
	\item Only be used for the stated purpose
	\item Be kept securely
	\item Handled in line with people's rights
	\item Not be transferred to countries without protection laws
\end{enumerate}

\enquote{Personal details} refers to living, identifiable people. This act includes paper and digital records. Exceptions are: \begin{itemize}
	\item National security, like data about suspected terrorists
	\item Crime and taxation, like policy surveillance
	\item Domestic purposes, like an address book
\end{itemize}

This is not foolproof, however. Companies experience data breaches on a semi-regular basis and the results are not good for the customers.

\sectitle{Computer Misuse Act 1990\label{computer-misuse-act}}

This act makes it an offence to access or modify computer material without permission. It makes \enquote{hacking} a crime.

It covers: \begin{itemize}
	\item Unauthorized access to computer material
	\item Unauthorized access with intent to commit or facilitate a crime
	\item Unauthorized modification of computer material
	\item Making, supplying, or obtaining anything which can be used in computer misuse offences
\end{itemize}

Examples include: \begin{itemize}
	\item Making or intentionally spreading a virus
	\item Attempting to login without authorization
	\item Using someone else's login
	\item Reading, changing, or deleting data without permission
	\item Obtaining or creating a \enquote{packet sniffer}
\end{itemize}

\sectitle{Copyright, Designs and Patents Act 1988\label{copyright-designs-and-patents-act}}

This protects creators of books, music, video, and software from having their work illegally copied. It applies to all forms of copying.

Digital storage hardware is very small and efficient, and fast broadband means that copies can be shared around the world very quickly. It is very easy to spread copies of digital media.

The software industry can take some steps to prevent illegal copying of software. For example: \begin{itemize}
	\item The user must enter a unique key before the software is installed
	\item Some software will only run is the CD is present in the drive
	\item Some applications will only run if a dongle is plugged in
	\item Some applications have always-online-DRM, which means they need a continuous internet connection
\end{itemize}

Tools used to create software may require fees if the software is then sold. Applications, games, books, films, and music are all protected, but algorithms cannot be copyrighted.

\sectitle{Regulation of Investigatory Powers Act 2000\label{regulation-of-investigatory-powers-act}}

This act: \begin{itemize}
	\item Requires ISPs to secretly assist in surveillance
	\item Enables mass surveillance of communications in transit and monitoring of internet activates
	\item Enables certain pubic bodies to demand that someone hand over keys to protected information
	\item Prevents the existence of interception warrants and any data collected with them from being revealed in court
\end{itemize}

As technology develops, laws may change. The UK Government has proposed an Investigatory Powers Bill to deal with interception of communication and acquiring bulk personal data. It's not good.

\end{document}
