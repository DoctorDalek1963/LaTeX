\documentclass[../main.tex]{subfile}

\begin{document}

\topictitle{Privacy and censorship}

\sectitle{Free speech}

The UK Human Rights Act protects free speech - but there are many exceptions such as incitement to racial or religious hatred, encouragement of terrorism, \enquote{Official Secrets}, and aspects of court proceedings. Films and video games are age-rated by non-governmental bodies. Court injunctions can prevent stories being printed, broadcast, or otherwise published.

Some countries make it illegal to criticise the Government or its leaders. Is it right to censor opinion? Is it right to censor anything? If so, who decides what to censor?

China, as well as other countries, regulates what information is available on Chinese internet.

In 2014, there were early 2000 convictions for threatening, offensive, or indecent messages, including tweets.

Twitter is US based and obviously not subject to UK law, but people in the UK who tweet do have to keep within UK law.

The volume of traffic makes monitoring impractical: it is up to users to report, block, unfollow, or take legal action against abusers. Some Facebook groups and newspaper comment forums are moderated. Moderation is about not letting anyone's agenda ruin the conversation or rant about irrelevant issues, as well as blocking trolls.

Twitter \textbf{did} manage to keep control of copyrighted TV footage of the 2016 footage from NBC, but they \textbf{don't} bother to keep control of harassment.

\sectitle{Monitoring behaviour}

Internet services such as Google are free at the point of use, but Google's main business model is advertising, and they use user data to target their adverts. Scott McNealy, chairman of Sun Microsystems said \enquote*{You have zero privacy anyway. Get over it.} Is he right? Should we care?

In 2011, the Association of British Insurers warned people \enquote*{not to disclose their summer holiday plans online, as criminals are increasingly going online to target unoccupied homes.} A McAfee Labs report found stolen credit cards with full supporting customer details on sale for \$30 - \$45.

Digital media makes distributing copies much easier. Videos, music, and software are all attractive targets for pirates.

\sectitle{Design, colour, and layout}

Colours have different meanings in different cultures. Different languages read text in different directions, so encoding the direction of the text is important. Thankfully, this is handled quite well by Unicode. However, things like tables and charts should match the primary direction of text as well. Unicode should be used everywhere in modern design - UTF-8 is best for western languages, but UTF-16 might be better for non-western languages.

\sectitle{Summary}

It is very hard to censor the internet. Governments, advertisers, and criminals may all be watching our behaviour. Digital technology makes piracy easy. It is tempting to be more offensive online than you would be in person. Layout, colour paradigms, and character sets are all affected by cultural expectations and traditions.

\end{document}
