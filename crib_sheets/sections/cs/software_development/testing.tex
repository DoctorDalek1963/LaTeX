\documentclass[../main.tex]{subfile}

\begin{document}

\topictitle{Testing}

\sectitle{Types of testing}

\textbf{Black box} testing is \textbf{integration} testing - independent of code. It looks at the program specification and creates a set of test data that covers all reasonable inputs, outputs, and program functions.

\textbf{White box} testing is \textbf{unit} testing - it depends on the details of the code logic. Unit tests should hit every code path at least once

\textbf{Alpha} testing is carried out by the software developer's in-house team. It can reveal errors or omissions in the definition of the system requirements and find bugs as early as possible. Alpha testing can also be done by calling the user in and asking them to evaluate the application at that stage of development. 

\textbf{Beta} testing is used when commercial software is being developed. The software is given to a subset of potential users, who agree to use the software and report any faults. Real users will do things that the developers never anticipated.

\textbf{Acceptance} testing is \textbf{evaluation}. The users needs to test every aspect of the software to make sure it does what it is supposed to do. It will be evaluated against the original specification document.

\sectitle{Types of maintenance}

\textbf{Corrective} maintenance: Bugs will usually be found when the software is used in the field, no matter how thoroughly the software was tested.

\textbf{Adaptive} maintenance: Over time, requirements will change and the software will have to be adapted to meet new needs.

\textbf{Perfective} maintenance: Even if the software works as intended, there are often ways of improving it. Perhaps by making it easier to use, faster, or adding quality-of-life improvements.

\end{document}
