\documentclass[../main.tex]{subfile}

\begin{document}

\topictitle{Minimum Spanning Trees}

A minimum spanning tree (MST) is a subgraph with contains all nodes but no cycles, and has the minimum total length.

\sectitle{Kruskal's algorithm}

An algorithm to find an MST is Kruskal's algorithm. It goes as follows:

\begin{enumerate}
	\item Sort all the edges into order of ascending weight
	\item Select the edge of least weight to start the tree
	\item\label{mst:kruskal:add-new-edge} Consider the next shortest edge. If it would form a cycle, then reject it, else add it to the tree
	\item Repeat step \ref{mst:kruskal:add-new-edge} until all nodes have been added
\end{enumerate}

\sectitle{Prim's algorithm}

Another algorithm to find an MST is Prim's algorithm. It goes as follows:

\begin{enumerate}
	\item Choose an arbitrary vertex to start the tree
	\item\label{mst:prim:add-new-edge} Select the edge of least weight which joins a vertex in the tree to a vertex not yet in the tree
	\item Repeat step \ref{mst:prim:add-new-edge} until all nodes have been added
\end{enumerate}

\sectitle{Prim's algorithm on a distance matrix}

To apply Prim's algorithm to a distance matrix, use the following algorithm:

\begin{enumerate}
	\item Choose an arbitrary vertex to start the tree
	\item\label{mst:prim-matrix:delete-row} Delete the \textbf{row} of the matrix for the chosen vertex
	\item\label{mst:prim-matrix:number-column} Number the \textbf{column} of the matrix for the chosen vertex
	\item\label{mst:prim-matrix:ring-entry} Put a ring around the smallest undeleted entry in the numbered columns
	\item\label{mst:prim-matrix:next-edge} The ringed entry becomes the next edge to join the tree
	\item Repeat steps \ref{mst:prim-matrix:delete-row} to \ref{mst:prim-matrix:next-edge} until all rows have been deleted
\end{enumerate}

\end{document}
