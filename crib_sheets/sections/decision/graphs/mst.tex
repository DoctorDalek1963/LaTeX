\documentclass[../main.tex]{subfile}

\begin{document}

\topictitle{Minimum Spanning Trees}

A minimum spanning tree (MST) is a subgraph with contains all nodes but no cycles, and has the minimum total length.

Throughout this MST topic, we will use the following graph for all our examples:

\begin{figure}[H]
\hspace{0.03\linewidth}
\begin{minipage}{0.45\linewidth}
	\begin{figure}[H]
		\centering
		\begin{tikzpicture}[scale=2]
			\node [dot, label=below left:$A$] (A) at (0, 0) {};
			\node [dot, label=below right:$B$] (B) at (1.5, 0.2) {};
			\node [dot, label=right:$C$] (C) at (2.7, 0.8) {};
			\node [dot, label=above:$D$] (D) at (0.75, 1.5) {};
			\node [dot, label=left:$E$] (E) at (-1.2, 0.7) {};

			\draw (A) -- node[below] {$7$} (B);
			\draw (A) -- node[above left] {$6$} (D);
			\draw (A) -- node[below left] {$5$} (E);
			\draw (B) -- node[above right] {$6$} (D);
			\draw (B) -- node[below right] {$5$} (C);
			\draw (C) -- node[above] {$8$} (D);
			\draw (D) -- node[above] {$4$} (E);
		\end{tikzpicture}
		\label{fig:mst:graph}
		\caption{The graph $\mathbf{G}$}
	\end{figure}
\end{minipage}\hfill
\begin{minipage}{0.45\linewidth}
	\begin{figure}[H]
		\centering
		{\renewcommand{\arraystretch}{1.15}
		\begin{tabular}{c|c c c c c c}
			& A & B & C & D & E\\
			\hline
			A & - & 7 & - & 6 & 5\\
			B & 7 & - & 5 & 6 & -\\
			C & - & 5 & - & 8 & -\\
			D & 6 & 6 & 8 & - & 4\\
			E & 5 & - & - & 4 & -
		\end{tabular}}
		\label{fig:mst:matrix}
		\caption{The distance matrix of $\mathbf{G}$}
	\end{figure}
\end{minipage}
\hspace{0.03\linewidth}
\end{figure}

\sectitle{Kruskal's algorithm}

An algorithm to find an MST is Kruskal's algorithm. It goes as follows:

\begin{enumerate}
	\item Sort all the edges into order of ascending weight
	\item Select the edge of least weight to start the tree
	\item\label{mst:kruskal:add-new-edge} Consider the next shortest edge. If it would form a cycle, then reject it, else add it to the tree
	\item Repeat step \ref{mst:kruskal:add-new-edge} until all nodes have been added
\end{enumerate}

Using the graph $\mathbf{G}$ as our example, we order the edges to get $DE(4), AE(5), BC(5), AD(6), BD(6), AB(7), CD(8)$, so we start with $DE$.

\begin{center}
\begin{tikzpicture}[scale=2]
	\node [dot, label=below left:$A$] (A) at (0, 0) {};
	\node [dot, label=below right:$B$] (B) at (1.5, 0.2) {};
	\node [dot, label=right:$C$] (C) at (2.7, 0.8) {};
	\node [dot, label=above:$D$] (D) at (0.75, 1.5) {};
	\node [dot, label=left:$E$] (E) at (-1.2, 0.7) {};

	\draw (D) -- node[above] {$4$} (E);
\end{tikzpicture}
\end{center}

$AE$ won't form a cycle, so we can add it next.

\begin{center}
\begin{tikzpicture}[scale=2]
	\node [dot, label=below left:$A$] (A) at (0, 0) {};
	\node [dot, label=below right:$B$] (B) at (1.5, 0.2) {};
	\node [dot, label=right:$C$] (C) at (2.7, 0.8) {};
	\node [dot, label=above:$D$] (D) at (0.75, 1.5) {};
	\node [dot, label=left:$E$] (E) at (-1.2, 0.7) {};

	\draw (D) -- node[above] {$4$} (E);
	\draw (A) -- node[below left] {$5$} (E);
\end{tikzpicture}
\end{center}

$BC$ also won't form a cycle, so we can add it next.

\begin{center}
\begin{tikzpicture}[scale=2]
	\node [dot, label=below left:$A$] (A) at (0, 0) {};
	\node [dot, label=below right:$B$] (B) at (1.5, 0.2) {};
	\node [dot, label=right:$C$] (C) at (2.7, 0.8) {};
	\node [dot, label=above:$D$] (D) at (0.75, 1.5) {};
	\node [dot, label=left:$E$] (E) at (-1.2, 0.7) {};

	\draw (D) -- node[above] {$4$} (E);
	\draw (A) -- node[below left] {$5$} (E);
	\draw (B) -- node[below right] {$5$} (C);
\end{tikzpicture}
\end{center}

We can see here that the subgraph at this stage is not connected. This is fine, since the final MST will be all connected.

$AD$ \textbf{would} form a cycle, so we reject it. $BD$ will not form a cycle, so we add it next.

\begin{center}
\begin{tikzpicture}[scale=2]
	\node [dot, label=below left:$A$] (A) at (0, 0) {};
	\node [dot, label=below right:$B$] (B) at (1.5, 0.2) {};
	\node [dot, label=right:$C$] (C) at (2.7, 0.8) {};
	\node [dot, label=above:$D$] (D) at (0.75, 1.5) {};
	\node [dot, label=left:$E$] (E) at (-1.2, 0.7) {};

	\draw (D) -- node[above] {$4$} (E);
	\draw (A) -- node[below left] {$5$} (E);
	\draw (B) -- node[below right] {$5$} (C);
	\draw (B) -- node[above right] {$6$} (D);
\end{tikzpicture}
\end{center}

All the vertices are connected, so we've created an MST of $\mathbf{G}$.

\sectitle{Prim's algorithm}

Another algorithm to find an MST is Prim's algorithm. It goes as follows:

\begin{enumerate}
	\item Choose an arbitrary vertex to start the tree
	\item\label{mst:prim:add-new-edge} Select the edge of least weight which joins a vertex in the tree to a vertex not yet in the tree
	\item Repeat step \ref{mst:prim:add-new-edge} until all nodes have been added
\end{enumerate}

We will choose to start at $D$. The shortest edge from $D$ is $DE$, so we will add it.

\begin{center}
\begin{tikzpicture}[scale=2]
	\node [dot, label=below left:$A$] (A) at (0, 0) {};
	\node [dot, label=below right:$B$] (B) at (1.5, 0.2) {};
	\node [dot, label=right:$C$] (C) at (2.7, 0.8) {};
	\node [dot, label=above:$D$] (D) at (0.75, 1.5) {};
	\node [dot, label=left:$E$] (E) at (-1.2, 0.7) {};

	\draw (D) -- node[above] {$4$} (E);
\end{tikzpicture}
\end{center}

Now the shortest edge from the current subgraph is $EA$, so we will add it.

\begin{center}
\begin{tikzpicture}[scale=2]
	\node [dot, label=below left:$A$] (A) at (0, 0) {};
	\node [dot, label=below right:$B$] (B) at (1.5, 0.2) {};
	\node [dot, label=right:$C$] (C) at (2.7, 0.8) {};
	\node [dot, label=above:$D$] (D) at (0.75, 1.5) {};
	\node [dot, label=left:$E$] (E) at (-1.2, 0.7) {};

	\draw (D) -- node[above] {$4$} (E);
	\draw (A) -- node[below left] {$5$} (E);
\end{tikzpicture}
\end{center}

The shortest edge from the current subgraph is $DA$ or $DB$ but $DA$ would create a cycle, so we will add $DB$.

\begin{center}
\begin{tikzpicture}[scale=2]
	\node [dot, label=below left:$A$] (A) at (0, 0) {};
	\node [dot, label=below right:$B$] (B) at (1.5, 0.2) {};
	\node [dot, label=right:$C$] (C) at (2.7, 0.8) {};
	\node [dot, label=above:$D$] (D) at (0.75, 1.5) {};
	\node [dot, label=left:$E$] (E) at (-1.2, 0.7) {};

	\draw (D) -- node[above] {$4$} (E);
	\draw (A) -- node[below left] {$5$} (E);
	\draw (B) -- node[above right] {$6$} (D);
\end{tikzpicture}
\end{center}

Now the shortest edge from the current subgraph is $BC$, so we will add it.

\begin{center}
\begin{tikzpicture}[scale=2]
	\node [dot, label=below left:$A$] (A) at (0, 0) {};
	\node [dot, label=below right:$B$] (B) at (1.5, 0.2) {};
	\node [dot, label=right:$C$] (C) at (2.7, 0.8) {};
	\node [dot, label=above:$D$] (D) at (0.75, 1.5) {};
	\node [dot, label=left:$E$] (E) at (-1.2, 0.7) {};

	\draw (D) -- node[above] {$4$} (E);
	\draw (A) -- node[below left] {$5$} (E);
	\draw (B) -- node[above right] {$6$} (D);
	\draw (B) -- node[below right] {$5$} (C);
\end{tikzpicture}
\end{center}

\sectitle{Prim's algorithm on a distance matrix}

To apply Prim's algorithm to a distance matrix, use the following algorithm:

\begin{enumerate}
	\item Choose an arbitrary vertex to start the tree
	\item\label{mst:prim-matrix:delete-row} Delete the \textbf{row} of the matrix for the chosen vertex
	\item\label{mst:prim-matrix:number-column} Number the \textbf{column} of the matrix for the chosen vertex
	\item\label{mst:prim-matrix:ring-entry} Put a ring around the smallest undeleted entry in the numbered columns
	\item\label{mst:prim-matrix:next-edge} The ringed entry becomes the next edge to join the tree
	\item Repeat steps \ref{mst:prim-matrix:delete-row} to \ref{mst:prim-matrix:next-edge} until all rows have been deleted
\end{enumerate}

We will do our example on the distance matrix of graph $\mathbf{G}$ and start at D. We will first delete its row and column, and then circle the smallest number in the numbered columns.

\begin{center}
	{\renewcommand{\arraystretch}{1.15}
	\begin{tabular}{c|c c c c c c}
		\multicolumn{1}{c}{} & & & & 1 &\\
		& A & B & C & D & E\\
		\hline
		A & - & 7 & - & 6 & 5\\
		B & 7 & - & 5 & 6 & -\\
		C & - & 5 & - & 8 & -\\
		D & 6 & 6 & 8 & - & 4\\[-1.8ex]
		\hline\noalign{\vspace{\dimexpr 1.8ex-\doublerulesep}}
		E & 5 & - & - & \cir{4} & -
	\end{tabular}}
\end{center}

We will now delete the E row and find the smallest number in the columns.

\begin{center}
	{\renewcommand{\arraystretch}{1.15}
	\begin{tabular}{c|c c c c c c}
		\multicolumn{1}{c}{} & & & & 1 & 2\\
		& A & B & C & D & E\\
		\hline
		A & - & 7 & - & 6 & \cir{5}\\
		B & 7 & - & 5 & 6 & -\\
		C & - & 5 & - & 8 & -\\
		D & 6 & 6 & 8 & - & 4\\[-1.8ex]
		\hline\noalign{\vspace{\dimexpr 1.8ex-\doublerulesep}}
		E & 5 & - & - & \cir{4} & -\\[-1.8ex]
		\hline\noalign{\vspace{\dimexpr 1.8ex-\doublerulesep}}
	\end{tabular}}
\end{center}

Now we will delete the A row and find the smallest number in the columns.

\begin{center}
	{\renewcommand{\arraystretch}{1.15}
	\begin{tabular}{c|c c c c c c}
		\multicolumn{1}{c}{} & 3 & & & 1 & 2\\
		& A & B & C & D & E\\
		\hline
		A & - & 7 & - & 6 & \cir{5}\\[-1.8ex]
		\hline\noalign{\vspace{\dimexpr 1.8ex-\doublerulesep}}
		B & 7 & - & 5 & \cir{6} & -\\
		C & - & 5 & - & 8 & -\\
		D & 6 & 6 & 8 & - & 4\\[-1.8ex]
		\hline\noalign{\vspace{\dimexpr 1.8ex-\doublerulesep}}
		E & 5 & - & - & \cir{4} & -\\[-1.8ex]
		\hline\noalign{\vspace{\dimexpr 1.8ex-\doublerulesep}}
	\end{tabular}}
\end{center}

Here, the smallest number in any of the numbered columns is the circled 6 above. When we look for the smallest number, we have to look at \textit{all} the numbered columns.

We now delete the B row and, again, find the smallest number.

\begin{center}
	{\renewcommand{\arraystretch}{1.15}
	\begin{tabular}{c|c c c c c c}
		\multicolumn{1}{c}{} & 3 & 4 & & 1 & 2\\
		& A & B & C & D & E\\
		\hline
		A & - & 7 & - & 6 & \cir{5}\\[-1.8ex]
		\hline\noalign{\vspace{\dimexpr 1.8ex-\doublerulesep}}
		B & 7 & - & 5 & \cir{6} & -\\[-1.8ex]
		\hline\noalign{\vspace{\dimexpr 1.8ex-\doublerulesep}}
		C & - & \cir{5} & - & 8 & -\\
		D & 6 & 6 & 8 & - & 4\\[-1.8ex]
		\hline\noalign{\vspace{\dimexpr 1.8ex-\doublerulesep}}
		E & 5 & - & - & \cir{4} & -\\[-1.8ex]
		\hline\noalign{\vspace{\dimexpr 1.8ex-\doublerulesep}}
	\end{tabular}}
\end{center}

We can now delete the final row and then we're done. The numbers on top of the columns show what order we added edges in.

\begin{center}
	{\renewcommand{\arraystretch}{1.15}
	\begin{tabular}{c|c c c c c c}
		\multicolumn{1}{c}{} & 3 & 4 & 5 & 1 & 2\\
		& A & B & C & D & E\\
		\hline
		A & - & 7 & - & 6 & \cir{5}\\[-1.8ex]
		\hline\noalign{\vspace{\dimexpr 1.8ex-\doublerulesep}}
		B & 7 & - & 5 & \cir{6} & -\\[-1.8ex]
		\hline\noalign{\vspace{\dimexpr 1.8ex-\doublerulesep}}
		C & - & \cir{5} & - & 8 & -\\[-1.8ex]
		\hline\noalign{\vspace{\dimexpr 1.8ex-\doublerulesep}}
		D & 6 & 6 & 8 & - & 4\\[-1.8ex]
		\hline\noalign{\vspace{\dimexpr 1.8ex-\doublerulesep}}
		E & 5 & - & - & \cir{4} & -\\[-1.8ex]
		\hline\noalign{\vspace{\dimexpr 1.8ex-\doublerulesep}}
	\end{tabular}}
\end{center}

\end{document}
