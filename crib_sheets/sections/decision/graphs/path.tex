\documentclass[../main.tex]{subfile}

\begin{document}

\topictitle{Shortest Path}

Throughout this shortest path topic, we will use the following graph for all our examples:

\begin{figure}[H]
\centering
\begin{tikzpicture}[scale=2.5]
	\node [dot, label=left:$A$] (A) at (0, 0) {};
	\node [dot, label=above:$B$] (B) at (1.2, 0) {};
	\node [dot, label=below:$C$] (C) at (1.2, -0.9) {};
	\node [dot, label=above right:$D$] (D) at (2.8, 0.8) {};
	\node [dot, label=above:$E$] (E) at (1.2, 1) {};
	\node [dot, label=right:$F$] (F) at (4, 0) {};

	\draw (A) -- node[above left] {$5$} (E);
	\draw (A) -- node[above] {$6$} (B);
	\draw (A) -- node[below left] {$2$} (C);
	\draw (E) -- node[above right] {$4$} (D);
	\draw (B) -- node[right] {$2$} (C);
	\draw (B) -- node[above left] {$4$} (D);
	\draw (B) -- node[above] {$8$} (F);
	\draw (C) -- node[below right] {$12$} (F);
	\draw (D) -- node[above right] {$3$} (F);
\end{tikzpicture}
\label{fig:path:graph}
\caption{The graph $\mathbf{G}$}
\end{figure}

\sectitle{Dijkstra's algorithm}

Dijkstra's algorithm can be used to find the shortest path between two chosen vertices in a network.

Dijkstra uses boxes like this in place of nodes to make the working clear:

\begin{center}
\begin{tabular}{|c|c|c|}
	\hline
	Vertex name & Order of labelling & Final label\\
	\hline
	\multicolumn{3}{|c|}{Working values}\\
	\hline
\end{tabular}
\end{center}

To find the shortest path from $S$ to $T$, use the following algorithm:

\begin{enumerate}
	\item Label the start vertex, $S$ with the final label 0
	\item\label{path:dijkstra:iterate-working-values} The vertex which just received its final label is $X$. For every unlabelled vertex $Y$ which is connected to $X$: \begin{enumerate}
			\item The new working value of $Y$ is the final label of $X$ plus the distance between $X$ and $Y$
			\item But only replace $Y$'s working value if the new one is smaller
	\end{enumerate}
	\item\label{path:dijkstra:choose-new-vertex} Look at all vertices without final labels. Choose the one with the smallest working value. This becomes the final label for this vertex
	\item Repeat steps \ref{path:dijkstra:iterate-working-values} and \ref{path:dijkstra:choose-new-vertex} until the destination vertex $T$ receives a final label
	\item To find the shortest path, trace back from $T$ to $S$. Include an edge only when its weight is the difference of the final labels of the vertices on each end of it
\end{enumerate}

As an example, lets find the shortest path from $A$ to $F$ on graph $\mathbf{G}$ using Dijkstra's algorithm.

We first redraw the graph with the boxes and fill in box $A$ with a final label of 0. This is the first box we labelled, so it gets an order number of 1.

\begin{figure}[H]
\centering
\begin{tikzpicture}[scale=2.5]
	\coordinate (A) at (0, 0);
	\coordinate (B) at (1.2, 0);
	\coordinate (C) at (1.2, -0.9);
	\coordinate (D) at (2.8, 0.8);
	\coordinate (E) at (1.2, 1);
	\coordinate (F) at (4, 0);

	\draw (A) -- node[above left] {$5$} (E);
	\draw (A) -- node[above] {$6$} (B);
	\draw (A) -- node[below left] {$2$} (C);
	\draw (E) -- node[above right] {$4$} (D);
	\draw (B) -- node[right] {$2$} (C);
	\draw (B) -- node[above left] {$4$} (D);
	\draw (B) -- node[above] {$8$} (F);
	\draw (C) -- node[below right] {$12$} (F);
	\draw (D) -- node[above right] {$3$} (F);

	\pic [at=(A)] {dijkbox={$A$}{1}{0}{}};
	\pic [at=(B)] {dijkbox={$B$}{}{}{}};
	\pic [at=(C)] {dijkbox={$C$}{}{}{}};
	\pic [at=(D)] {dijkbox={$D$}{}{}{}};
	\pic [at=(E)] {dijkbox={$E$}{}{}{}};
	\pic [at=(F)] {dijkbox={$F$}{}{}{}};
\end{tikzpicture}
\end{figure}

We now fill in the working values for all visit-able nodes and thus choose $C$ as the next node to visit, so it gets a final label.

\begin{figure}[H]
\centering
\begin{tikzpicture}[scale=2.5]
	\coordinate (A) at (0, 0);
	\coordinate (B) at (1.2, 0);
	\coordinate (C) at (1.2, -0.9);
	\coordinate (D) at (2.8, 0.8);
	\coordinate (E) at (1.2, 1);
	\coordinate (F) at (4, 0);

	\draw (A) -- node[above left] {$5$} (E);
	\draw (A) -- node[above] {$6$} (B);
	\draw (A) -- node[below left] {$2$} (C);
	\draw (E) -- node[above right] {$4$} (D);
	\draw (B) -- node[right] {$2$} (C);
	\draw (B) -- node[above left] {$4$} (D);
	\draw (B) -- node[above] {$8$} (F);
	\draw (C) -- node[below right] {$12$} (F);
	\draw (D) -- node[above right] {$3$} (F);

	\pic [at=(A)] {dijkbox={$A$}{1}{0}{}};
	\pic [at=(B)] {dijkbox={$B$}{}{}{6}};
	\pic [at=(C)] {dijkbox={$C$}{2}{2}{2}};
	\pic [at=(D)] {dijkbox={$D$}{}{}{}};
	\pic [at=(E)] {dijkbox={$E$}{}{}{5}};
	\pic [at=(F)] {dijkbox={$F$}{}{}{}};
\end{tikzpicture}
\end{figure}

We now repeat that process and choose $B$ as the next node to visit, so it gets a final label.

\begin{figure}[H]
\centering
\begin{tikzpicture}[scale=2.5]
	\coordinate (A) at (0, 0);
	\coordinate (B) at (1.2, 0);
	\coordinate (C) at (1.2, -0.9);
	\coordinate (D) at (2.8, 0.8);
	\coordinate (E) at (1.2, 1);
	\coordinate (F) at (4, 0);

	\draw (A) -- node[above left] {$5$} (E);
	\draw (A) -- node[above] {$6$} (B);
	\draw (A) -- node[below left] {$2$} (C);
	\draw (E) -- node[above right] {$4$} (D);
	\draw (B) -- node[right] {$2$} (C);
	\draw (B) -- node[above left] {$4$} (D);
	\draw (B) -- node[above] {$8$} (F);
	\draw (C) -- node[below right] {$12$} (F);
	\draw (D) -- node[above right] {$3$} (F);

	\pic [at=(A)] {dijkbox={$A$}{1}{0}{}};
	\pic [at=(B)] {dijkbox={$B$}{3}{4}{\cancel{6} 4}};
	\pic [at=(C)] {dijkbox={$C$}{2}{2}{2}};
	\pic [at=(D)] {dijkbox={$D$}{}{}{}};
	\pic [at=(E)] {dijkbox={$E$}{}{}{5}};
	\pic [at=(F)] {dijkbox={$F$}{}{}{14}};
\end{tikzpicture}
\end{figure}

We then repeat that process once again and choose $E$ as the next node.

\begin{figure}[H]
\centering
\begin{tikzpicture}[scale=2.5]
	\coordinate (A) at (0, 0);
	\coordinate (B) at (1.2, 0);
	\coordinate (C) at (1.2, -0.9);
	\coordinate (D) at (2.8, 0.8);
	\coordinate (E) at (1.2, 1);
	\coordinate (F) at (4, 0);

	\draw (A) -- node[above left] {$5$} (E);
	\draw (A) -- node[above] {$6$} (B);
	\draw (A) -- node[below left] {$2$} (C);
	\draw (E) -- node[above right] {$4$} (D);
	\draw (B) -- node[right] {$2$} (C);
	\draw (B) -- node[above left] {$4$} (D);
	\draw (B) -- node[above] {$8$} (F);
	\draw (C) -- node[below right] {$12$} (F);
	\draw (D) -- node[above right] {$3$} (F);

	\pic [at=(A)] {dijkbox={$A$}{1}{0}{}};
	\pic [at=(B)] {dijkbox={$B$}{3}{4}{\cancel{6} 4}};
	\pic [at=(C)] {dijkbox={$C$}{2}{2}{2}};
	\pic [at=(D)] {dijkbox={$D$}{}{}{8}};
	\pic [at=(E)] {dijkbox={$E$}{4}{5}{5}};
	\pic [at=(F)] {dijkbox={$F$}{}{}{\cancel{14} 12}};
\end{tikzpicture}
\end{figure}

We choose $D$ next, since $E$ offers no new lengths.

\begin{figure}[H]
\centering
\begin{tikzpicture}[scale=2.5]
	\coordinate (A) at (0, 0);
	\coordinate (B) at (1.2, 0);
	\coordinate (C) at (1.2, -0.9);
	\coordinate (D) at (2.8, 0.8);
	\coordinate (E) at (1.2, 1);
	\coordinate (F) at (4, 0);

	\draw (A) -- node[above left] {$5$} (E);
	\draw (A) -- node[above] {$6$} (B);
	\draw (A) -- node[below left] {$2$} (C);
	\draw (E) -- node[above right] {$4$} (D);
	\draw (B) -- node[right] {$2$} (C);
	\draw (B) -- node[above left] {$4$} (D);
	\draw (B) -- node[above] {$8$} (F);
	\draw (C) -- node[below right] {$12$} (F);
	\draw (D) -- node[above right] {$3$} (F);

	\pic [at=(A)] {dijkbox={$A$}{1}{0}{}};
	\pic [at=(B)] {dijkbox={$B$}{3}{4}{\cancel{6} 4}};
	\pic [at=(C)] {dijkbox={$C$}{2}{2}{2}};
	\pic [at=(D)] {dijkbox={$D$}{5}{8}{8}};
	\pic [at=(E)] {dijkbox={$E$}{4}{5}{5}};
	\pic [at=(F)] {dijkbox={$F$}{}{}{\cancel{14} 12}};
\end{tikzpicture}
\end{figure}

And again.

\begin{figure}[H]
\centering
\begin{tikzpicture}[scale=2.5]
	\coordinate (A) at (0, 0);
	\coordinate (B) at (1.2, 0);
	\coordinate (C) at (1.2, -0.9);
	\coordinate (D) at (2.8, 0.8);
	\coordinate (E) at (1.2, 1);
	\coordinate (F) at (4, 0);

	\draw (A) -- node[above left] {$5$} (E);
	\draw (A) -- node[above] {$6$} (B);
	\draw (A) -- node[below left] {$2$} (C);
	\draw (E) -- node[above right] {$4$} (D);
	\draw (B) -- node[right] {$2$} (C);
	\draw (B) -- node[above left] {$4$} (D);
	\draw (B) -- node[above] {$8$} (F);
	\draw (C) -- node[below right] {$12$} (F);
	\draw (D) -- node[above right] {$3$} (F);

	\pic [at=(A)] {dijkbox={$A$}{1}{0}{}};
	\pic [at=(B)] {dijkbox={$B$}{3}{4}{\cancel{6} 4}};
	\pic [at=(C)] {dijkbox={$C$}{2}{2}{2}};
	\pic [at=(D)] {dijkbox={$D$}{5}{8}{8}};
	\pic [at=(E)] {dijkbox={$E$}{4}{5}{5}};
	\pic [at=(F)] {dijkbox={$F$}{6}{11}{\cancel{14}\cancel{12}\,11}};
\end{tikzpicture}
\end{figure}

The target node $F$ now has a final label, so we have finished the algorithm. The shortest distance from $A$ to $F$ is 11 and to find the path, we backtrack from $F$ to $A$.

\begin{figure}[H]
\centering
\begin{tikzpicture}[scale=2.5]
	\coordinate (A) at (0, 0);
	\coordinate (B) at (1.2, 0);
	\coordinate (C) at (1.2, -0.9);
	\coordinate (D) at (2.8, 0.8);
	\coordinate (E) at (1.2, 1);
	\coordinate (F) at (4, 0);

	\draw (A) -- node[above left] {$5$} (E);
	\draw (A) -- node[above] {$6$} (B);
	\draw[ultra thick] (A) -- node[below left] {$2$} (C);
	\draw (E) -- node[above right] {$4$} (D);
	\draw[ultra thick] (B) -- node[right] {$2$} (C);
	\draw[ultra thick] (B) -- node[above left] {$4$} (D);
	\draw (B) -- node[above] {$8$} (F);
	\draw (C) -- node[below right] {$12$} (F);
	\draw[ultra thick] (D) -- node[above right] {$3$} (F);

	\pic [at=(A)] {dijkbox={$A$}{1}{0}{}};
	\pic [at=(B)] {dijkbox={$B$}{3}{4}{\cancel{6} 4}};
	\pic [at=(C)] {dijkbox={$C$}{2}{2}{2}};
	\pic [at=(D)] {dijkbox={$D$}{5}{8}{8}};
	\pic [at=(E)] {dijkbox={$E$}{4}{5}{5}};
	\pic [at=(F)] {dijkbox={$F$}{6}{11}{\cancel{14}\cancel{12}\,11}};
\end{tikzpicture}
\end{figure}

The final labels represent the distance between the start vertex and that labelled vertex, and you can use the backtracking method to find the shortest path between the start node and any vertex with a final label. To find the shortest path between a different pair of vertices, you'll have to do Dijkstra's again. Or you could use Floyd's.

\sectitle{Floyd's algorithm}

Floyd's algorithm can be used to find the shortest path between every pair of vertices in a network. It produces two tables - the first shows the minimum distance between any two vertices, and the second can be used to find the route.

\begin{enumerate}
	\item Complete an initial \textbf{distance table} for the network. If there is no direct route from one vertex to another, label the distance with $\infty$
	\item Complete an initial \textbf{route table} by making every entry in a column the same as the label at the top of the column
	\item For the first iteration, copy the values in the first row and column in the tables into a new pair of tables. Shade these values
	\item\label{path:floyd:iterate-unshaded} Consider each unshaded position in turn: \begin{enumerate}
		\item Compare the value in this position in the previous table to the sum of the corresponding shaded values. Choose the smaller of the two options
		\item If you chose the sum, then copy the letter of this vertex into the corresponding position in the route table
		\item Repeat until all unshaded cells have been filled in
	\end{enumerate}
	\item For the second iteration, copy the second row and column from the previous tables into a new pair of tables and shade them. Repeat step \ref{path:floyd:iterate-unshaded}
	\item Continue iterating like this until you've done every row and column
\end{enumerate}

\end{document}
