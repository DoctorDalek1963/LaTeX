\documentclass[../main.tex]{subfile}

\begin{document}

\topictitle{Shortest Path}

\sectitle{Dijkstra's algorithm}

Dijkstra's algorithm can be used to find the shortest path between two chosen vertices in a network.

Dijkstra uses boxes like this in place of nodes to make the working clear:

\begin{center}
\begin{tabular}{|c|c|c|}
	\hline
	Vertex name & Order of labelling & Final label\\
	\hline
	\multicolumn{3}{|c|}{Working values}\\
	\hline
\end{tabular}
\end{center}

To find the shortest path from $S$ to $T$, use the following algorithm:

\begin{enumerate}
	\item Label the start vertex, $S$ with the final label 0
	\item\label{path:dijkstra:iterate-working-values} The vertex which just received its final label is $X$. For every unlabelled vertex $Y$ which is connected to $X$: \begin{enumerate}
			\item The new working value of $Y$ is the final label of $X$ plus the distance between $X$ and $Y$
			\item But only replace $Y$'s working value if the new one is smaller
	\end{enumerate}
	\item\label{path:dijkstra:choose-new-vertex} Look at all vertices without final labels. Choose the one with the smallest working value. This becomes the final label for this vertex
	\item Repeat steps \ref{path:dijkstra:iterate-working-values} and \ref{path:dijkstra:choose-new-vertex} until the destination vertex $T$ receives a final label
	\item To find the shortest path, trace back from $T$ to $S$. Include an edge only when its weight is the difference of the final labels of the vertices on each end of it
\end{enumerate}

\sectitle{Floyd's algorithm}

Floyd's algorithm can be used to find the shortest path between every pair of vertices in a network. It produces two tables - the first shows the minimum distance between any two vertices, and the second can be used to find the route.

\begin{enumerate}
	\item Complete an initial \textbf{distance table} for the network. If there is no direct route from one vertex to another, label the distance with $\infty$
	\item Complete an initial \textbf{route table} by making every entry in a column the same as the label at the top of the column
	\item For the first iteration, copy the values in the first row and column in the tables into a new pair of tables. Shade these values
	\item\label{path:floyd:iterate-unshaded} Consider each unshaded position in turn: \begin{enumerate}
		\item Compare the value in this position in the previous table to the sum of the corresponding shaded values. Choose the smaller of the two options
		\item If you chose the sum, then copy the letter of this vertex into the corresponding position in the route table
		\item Repeat until all unshaded cells have been filled in
	\end{enumerate}
	\item For the second iteration, copy the second row and column from the previous tables into a new pair of tables and shade them. Repeat step \ref{path:floyd:iterate-unshaded}
	\item Continue iterating like this until you've done every row and column
\end{enumerate}

\end{document}
