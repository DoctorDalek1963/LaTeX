\documentclass[../main.tex]{subfile}

\begin{document}

\topictitle{Probability Distributions}

\sectitle{Poisson}

The Poisson distribution is used to model a situation where an event occurs at a fixed rate.

You can model $X$ as a Poisson distribution if:
\begin{itemize}
	\item The events must occur independently
	\item They must occur singly in space or time
	\item The events must occur at a constant average rate
\end{itemize}

\begin{center}
	If $X \sim \text{Po}(\lambda)$, then $\boxed{P(X = x) = \frac{e^{-\lambda} \lambda^x}{x!}}\ \constraint{x \geq 0}$
\end{center}

\sectitle{Geometric}

The Geometric distribution is used to model a situation where you try an event several times until a success occurs, and you want to know how many tries it will take.

You can model $X$ as a Geometric distribution if:
\begin{itemize}
	\item Each attempt is independent
	\item Each attempt has the same probability
\end{itemize}

\begin{center}
	If $X \sim \text{Geo}(p)$, then $\boxed{P(X = x) = p(1 - p)^{x - 1}}\ \constraint{x > 0}$

\end{center}

\begin{figure}[H]
\hspace{0.15\linewidth}
\begin{minipage}{0.3\linewidth}
\begin{itemize}
	\item $\boxed{P(X \leq x) = 1 - (1 - p)^x}$
	\item $P(X > x) = (1 - p)^x$
\end{itemize}
\end{minipage}\hfill
\begin{minipage}{0.3\linewidth}
\begin{itemize}
	\item $\boxed{P(X \geq x) = (1 - p)^{x - 1}}$
	\item $P(X < x) = 1 - (1 - p)^{x - 1}$
\end{itemize}
\end{minipage}
\hspace{0.15\linewidth}
\end{figure}

\end{document}
