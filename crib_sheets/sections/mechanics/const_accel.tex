\documentclass[../main.tex]{subfile}

\begin{document}

\topictitle{Constant Acceleration}

\sectitle{Formulae}

Constant acceleration (like free-fall under gravity) can be modelled using $suvat$. $s$ is displacement, $u$ is initial velocity, $v$ is final velocity, $a$ is acceleration, and $t$ is time taken. You almost always have one of these variables missing or not relevant, so use the equation that doesn't include that variable to find the one that you do care about.

\begin{empheq}[box=\formulaBookBox]{align*}
	s \implies& v = u + at\\
	u \implies& s = vt - \frac{1}{2} at^2\\
	v \implies& s = ut + \frac{1}{2} at^2\\
	a \implies& s = \frac{1}{2}\left( u + v \right) t\\
	t \implies& v^2 = u^2 + 2as
\end{empheq}

\begin{empheq}[box=\rememberBox]{gather*}
	\text{If a question doesn't specify the value of $g$, use 9.8 ms$^{-1}$}
\end{empheq}

\end{document}
