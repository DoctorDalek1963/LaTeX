\documentclass[../main.tex]{subfile}

\begin{document}

\topictitle{Forces}

\sectitle{Tackling a question}

When tackling a question, start by saying \enquote{\textbf{R}ight, \textbf{W}hat \textbf{A}re \textbf{T}he \textbf{F}orces?} This stands for \textbf{R}eaction forces, \textbf{W}eight, \textbf{A}cceleration, \textbf{T}ension, and \textbf{F}riction.

Then label all these things on a large, clear diagram. Imagine a situation like the following, where we have a block \textbf{P} of mass 10 kg on an inclined plane at angle $\theta$ where $\tan\theta = \frac{3}{4}$, with coefficient of friction $\mu$, attached via a pulley to a block \textbf{Q} of mass 2 kg. The block \textbf{P} is on the point of slipping down the slope. Find $\mu$.

\begin{center}
\begin{tikzpicture}
	\coordinate (O) at (0, 0);
	\coordinate (A) at (8, 6);
	\coordinate (B) at (8, 0);

	% Triangle
	\draw (O) -- (A) -- (B) -- cycle;
	% Pulley
	\draw [fill=white] (A) circle[radius=0.6];

	% Box P
	\pic [rotate={atan(3/4)}] at ({atan(3/4)}:4) {box={1.6}{P}};
	% Box Q
	\pic [rotate=270] at (8.2,2) {box={0.8}{Q}};

	\coordinate (P side) at ($ ({atan(3/4)}:{4+0.8}) + ({atan(3/4)+90}:0.6) $);
	\coordinate (pulley P side) at ($ (A) + ({atan(3/4)+90}:0.6) $);
	\coordinate (pulley Q side) at ($ (A) + (0.6, 0) $);
	\coordinate (Q top) at (8.6, 2.4);

	% Rope
	\draw (P side) -- (pulley P side)
		arc [start angle={90+atan(3/4)}, end angle={0}, radius=0.6]
		(pulley Q side) -- (Q top);

	% Angle
	\tkzMarkAngle(B,O,A);
	\tkzLabelAngle[pos=0.75](B,O,A){$\theta$};
\end{tikzpicture}
\end{center}

We can label this diagram with force arrows and other information using the mnemonic from above.

\begin{center}
\begin{tikzpicture}
	\coordinate (O) at (0, 0);
	\coordinate (A) at (8, 6);
	\coordinate (B) at (8, 0);
	\coordinate (P side) at ($ ({atan(3/4)}:{4+0.8}) + ({atan(3/4)+90}:0.6) $);
	\coordinate (pulley P side) at ($ (A) + ({atan(3/4)+90}:0.6) $);
	\coordinate (pulley Q side) at ($ (A) + (0.6, 0) $);
	\coordinate (Q top) at (8.6, 2.4);
	\coordinate (P centre) at ($ ({atan(3/4)}:4) + ({atan(3/4)+90}:0.8) $);
	\coordinate (Q centre) at (8.6, 2);

	\draw[force arrow] (P centre) -- +({atan(3/4)+90}:2) node[above right] {$R_\textbf{P}$};
	\draw[force arrow] (P centre) -- +(270:2) node[right] {$10g$};
	\draw[force arrow] (P side) -- +({atan(3/4)}:1.2) node[below, xshift=0.4em] {$T$};
	\draw[force arrow] (P side) ++({atan(3/4)+90}:0.4) -- +({atan(3/4)}:1.2) node[above] {$\mu R_\textbf{P}$};

	\draw[force arrow] (Q centre) -- +(0, 1.5) node[right] {$T$};
	\draw[force arrow] (Q centre) -- +(0, -1.5) node[right] {$2g$};

	\begin{scope} % Original diagram
		% Triangle
		\draw (O) -- (A) -- (B) -- cycle;
		% Pulley
		\draw [fill=white] (A) circle[radius=0.6];

		% Box P
		\pic [rotate={atan(3/4)}] at ({atan(3/4)}:4) {box={1.6}{P}};
		% Box Q
		\pic [rotate=270] at (8.2,2) {box={0.8}{Q}};

		% Rope
		\draw (P side) -- (pulley P side)
			arc [start angle={90+atan(3/4)}, end angle={0}, radius=0.6]
			(pulley Q side) -- (Q top);

		% Angle
		\tkzMarkAngle(B,O,A);
		\tkzLabelAngle[pos=0.75](B,O,A){$\theta$};
	\end{scope}
\end{tikzpicture}
\end{center}

We can then convert this diagram into free-body diagrams for \textbf{P} and \textbf{Q} and resolve the forces on \textbf{P} parallel and perpendicular to the plane.

\begin{center}
\begin{tikzpicture}
	\coordinate (P centre) at (0, 0);
	\coordinate (Q centre) at (5, 0);

	% Box P
	\draw[force arrow] (P centre) -- +({atan(3/4)+90}:2) node[above right] {$R_\textbf{P}$};
	\draw[force arrow] (P centre) -- +({atan(3/4)}:2) node[above] {$T + \mu R_\textbf{P}$};
	\draw[force arrow] (P centre) -- +({atan(3/4)+180}:2) node[below] {$10g\sin\theta$};
	\draw[force arrow] (P centre) -- +({atan(3/4)-90}:2) node[below] {$10g\cos\theta$};

	\pic [rotate={atan(3/4)}] at ($ (P centre) + ({atan(3/4)-90}:0.8) $) {box={1.6}{P}};

	% Box Q
	\draw[force arrow] (Q centre) -- +(0, 1.5) node[right] {$T$};
	\draw[force arrow] (Q centre) -- +(0, -1.5) node[right] {$2g$};

	\pic at ($ (Q centre) + (0, -0.4) $) {box={0.8}{Q}};
\end{tikzpicture}
\end{center}

We know that $\tan\theta = \frac{3}{4}$, so we can find $\sin\theta$ and $\cos\theta$. We also know that both blocks are at rest:

\begin{center}
\begin{tikzpicture}
	\coordinate (P centre) at (0, 0);
	\coordinate (Q centre) at (5, 0);

	% Box P
	\draw[force arrow] (P centre) -- +({atan(3/4)+90}:2) node[above right] {$R_\textbf{P}$};
	\draw[force arrow] (P centre) -- +({atan(3/4)}:2) node[above] {$T + \mu R_\textbf{P}$};
	\draw[force arrow] (P centre) -- +({atan(3/4)+180}:2) node[below] {$6g$};
	\draw[force arrow] (P centre) -- +({atan(3/4)-90}:2) node[below] {$8g$};

	\pic [rotate={atan(3/4)}] at ($ (P centre) + ({atan(3/4)-90}:0.8) $) {box={1.6}{P}};

	% Box Q
	\draw[force arrow] (Q centre) -- +(0, 1.5) node[right] {$T$};
	\draw[force arrow] (Q centre) -- +(0, -1.5) node[right] {$2g$};

	\pic at ($ (Q centre) + (0, -0.4) $) {box={0.8}{Q}};
\end{tikzpicture}
\end{center}

Thus, we get the following set of the equations:
\begin{eqnarray}
	R_\textbf{P} = 8g\\
	6g = T + 8g\mu\\
	T = 2g
\end{eqnarray}

From which we can easily derive $6g = g(2 + 8\mu) \implies 4 = 8\mu$ and find that $\mu = \frac{1}{2}$.

\end{document}
