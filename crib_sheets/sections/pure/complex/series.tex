\documentclass[../main.tex]{subfile}

\begin{document}

\topictitle{Sums of Series}

Questions in this topic give you a geometric series with complex terms. You have to use the geometric series formulae and then simplify the result.

Here's an example:

$$S = e^{i\theta} + e^{2i\theta} + e^{3i\theta} + \cdots + e^{8i\theta}\ \constraint{\theta \ne 2n\pi}$$

We want to condense the series into a single expression, so we start by applying the geometric series formula.

$$S_n = \dfrac{a(r^n - 1)}{r - 1}
\implies S = \frac{e^{i\theta} \big( (e^{i\theta})^8 - 1 \big)}{e^{i\theta} - 1}
= \frac{e^{i\theta} \big( e^{8i\theta} - 1 \big)}{e^{i\theta} - 1}$$

We're going to use Technique~\ref{technique:Denominators} to simplify this, so we multiply the top and bottom by $\displaystyle e^{-\frac{i\theta}{2}}$.

$$\frac{e^{\frac{i\theta}{2}} \big(e^{8i\theta} - 1\big)}{e^{\frac{i\theta}{2}} - e^{\frac{i\theta}{2}}}
= \frac{e^{\frac{i\theta}{2}} \big(e^{8i\theta} - 1\big)}{2i\sin \frac{\theta}{2}}$$

Now we can factorise the numerator using Technique~\ref{technique:Numerators} and simplify it.

$$\frac{e^{\frac{i\theta}{2}} e^{4i\theta} \big(e^{4i\theta} - e^{-4i\theta}\big)}{2i\sin \frac{\theta}{2}}
= \frac{e^{\frac{9i\theta}{2}} (2i\sin 4\theta)}{2i\sin \frac{\theta}{2}}
= \frac{e^{\frac{9i\theta}{2}} \sin 4\theta}{\sin \frac{\theta}{2}}$$

Now you could split the numerator into real and imaginary parts if the questions asked for it.

\sectitle{Techniques}

The aim of the process is normally to get something of the form $z^n \pm z^{-n}$.

\technique{Denominators}

$$\Large \boxed{\frac{\cdots}{e^{ni\theta} \pm 1}
= \frac{\cdots}{e^{ni\theta} \pm 1} \times \frac{e^\frac{-ni\theta}{2}}{e^\frac{-ni\theta}{2}}
= \frac{e^\frac{-ni\theta}{2} (\cdots)}{e^\frac{ni\theta}{2} \pm e^\frac{-ni\theta}{2}}}$$

When the denominator is of the form $\displaystyle e^{ni\theta} \pm 1$, we can multiply the top and bottom of the fraction by the negative half power of $e$. This gives us $z^n \pm z^{-n}$ in the denominator, which we can get rid of with ease.

\technique{Denominator General Case}

$$\Large \boxed{\frac{\cdots}{e^{ni\theta} + k}
= \frac{\cdots}{e^{ni\theta} + k} \times \frac{e^{-ni\theta} + k}{e^{-ni\theta} + k}
= \frac{(e^{-ni\theta} + k) (\cdots)}{(e^{ni\theta} + k) (e^{-ni\theta} + k)}}$$

Always factor the denominator to be of the form $e^{ni\theta} + k$. If $k = \pm 1$, then use Technique~\ref{technique:Denominators}. Otherwise, multiply the top and bottom by the bottom, but with the sign of the power flipped. The sign between terms stays the same, but the power of $e$ gets negated (and not halved).

The denominator will then expand to create something that can be simplified with little difficulty.

\technique{Numerators}

$$\Large \boxed{\frac{e^{ni\theta} \pm 1}{\cdots}
= \frac{e^\frac{ni\theta}{2} \big(e^\frac{ni\theta}{2} \pm e^\frac{-ni\theta}{2}\big)}{\cdots}}$$

Once the denominator is in a manageable form, we can then apply this technique to the numerator.

When you've got something of the form $\displaystyle e^{ni\theta} \pm 1$, you can factor out $e$ to half of the power to get $\displaystyle e^\frac{ni\theta}{2} \big(e^\frac{ni\theta}{2} \pm e^\frac{-ni\theta}{2}\big)$. The inside of the bracket is now of the form $z^n \pm z^{-n}$, so we can simplify it.

\end{document}
