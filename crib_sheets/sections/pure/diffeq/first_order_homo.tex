\documentclass[../main.tex]{subfile}

\begin{document}

\topictitle{First Order Homogeneous\label{topic:first-order-homogeneous}}

\sectitle{Separation of variables\label{sec:first-order-homo:separation-of-variables}}

The simplest technique for solving first order homogeneous differential equations is to separate the variables. This involves treating the $\diffd x$ and $\diffd y$ as their own variables, rearranging everything to get all $y$ terms on one side and all $x$ terms on the other, and then adding integral signs. For example,
\begin{gather*}
	\dd{y}{x} = -\frac{y}{x}
	\implies \frac{1}{y} \diffd y = -\frac{1}{x} \diffd x
	\implies \inte{\frac{1}{y}}{y} = \inte{-\frac{1}{x}}{x}\\[1.2ex]
	\implies \ln|y| = -\ln|x| + c
	\implies \ln|y| + \ln|x| = c
	\implies \ln|xy| = c\\[1.2ex]
	\implies xy = \pm A
	\implies y = \pm \frac{A}{x}
\end{gather*}

This technique is easy, but very very rarely possible.

\sectitle{Reverse product rule\label{sec:first-order-homo:reverse-product-rule}}

Sometimes the LHS (the $y$ and $y'$ terms in this case) can be seen as the result of the product rule applied to some function of $x$ and $y$. For example,
\begin{gather*}
	x^3 y' + 3x^2 y = \sin x
	\implies \diff{x} (x^3 y) = \sin x
	\implies x^3 y = \inte{\sin x}{x}\\[1.2ex]
	\implies x^3 y = -\cos x + c
	\implies y = \frac{-\cos x + c}{x^3}
\end{gather*}

Again, this technique is easy, but very rarely possible.

\sectitle{Integrating factor\label{sec:first-order-homo:integrating-factor}}

The reverse product rule is not always applicable, but you can always multiply a first order linear homogeneous differential equation by some function of $x$, called an \textbf{integrating factor (IF)}, to make the reverse product rule applicable.

This \textbf{IF} could be anything, and you may be able to find a simple one, but there is a general formula:
{\Large \begin{empheq}[box=\rememberBox]{gather*}
	y' + P(x)y = Q(x) \implies \textbf{IF} = e^{\int P(x) \diffd x}
\end{empheq}}

For example,
\begin{gather*}
	y' + \frac{3y}{x} = \frac{\sin x}{x^3}
	\implies \textbf{IF} = x^3\text{ (by inspection) or } \textbf{IF} = e^{\inte{\frac{3}{x}}{x}} = e^{3\ln|x|} = x^3\\[1.2ex]
	x^3 y' + 3x^2 y = \sin x
	\implies y = \frac{-\cos x + c}{x^3}\text{ as shown above}
\end{gather*}

\end{document}
