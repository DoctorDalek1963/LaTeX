\documentclass[../main.tex]{subfile}

\begin{document}

\topictitle{Exponentials and Logarithms}

\sectitle{Exponentials}

\begin{figure}[h]
\hspace{0.05\linewidth}
\begin{minipage}{0.55\linewidth}
	\begin{tikzpicture}
		\begin{axis}[
			default axis,
			samples=2200,
			xmin=-3.5,
			xmax=7.5,
			ymin=-3.5,
			ymax=7.5,
			restrict x to domain=-3.5:7.5,
			restrict y to domain=-3.5:7.5,
			width=\linewidth
		]
			\addplot [thick] {exp(x)};
			\addplot [dashed, domain=-3.5:7.5] {x};
			\addplot [thick, domain=0:7.5] {ln(x)};

			\coordinate (exp title) at (axis cs:2,7.5) {};
			\coordinate (ln title) at (axis cs:7.5,2) {};
			\coordinate (yex title) at (axis cs:7,7) {};
		\end{axis}

		\node at (exp title) [above] {$y = e^x$};
		\node at (ln title) [right] {$y = \ln x$};
		\node at (yex title) [below right] {$y = x$};
	\end{tikzpicture}
\end{minipage}\hfill
\begin{minipage}{0.39\linewidth}
\large
\begin{empheq}[box=\rememberBox]{align*}
	\diff{x} e^x &= e^x\\[2ex]
	\diff{x} e^{kx} &= k e^{kx}\\[2ex]
	\diff{x} e^{f(x)} &= f'(x) e^{f(x)}
\end{empheq}

\vspace{1em}
$\ln x$ is the inverse of $e^x$, meaning its graph is reflected in the line $y = x$.
\end{minipage}
\end{figure}

\sectitle{Log Laws}

\vspace{-3ex}
{\large \begin{empheq}[box=\formulaBookBox]{align*}
	\log_a b \equiv \frac{\ln b}{\ln a}
\end{empheq}}

\vspace{-1ex}
{\large \begin{empheq}[box=\rememberBox]{alignat*=3}
	\log xy &\equiv \log x + \log y\ \ \ \ \ &
		\log \frac{x}{y} &\equiv \log x - \log y\ \ \ \ \ &
		\log x^y &\equiv y \log x \\
	\log_a a &\equiv 1\ \ \ \ \ &
		\log 1 &\equiv 0\ \ \ \ \ &
		\log \frac{1}{x} &\equiv -\log x
\end{empheq}}

\sectitle{Log Plots}

\begin{figure}[H]
\begin{minipage}{0.49\linewidth}
	\begin{tikzpicture}
		\begin{axis}[
			grid=none,
			no marks,
			axis lines=middle,
			ticks=none,
			xmin=-1,
			xmax=10,
			ymin=-1,
			ymax=10,
			xlabel=$\log x$,
			ylabel=$\log y$,
			width=\linewidth
		]
			\addplot [thick, domain=-1:10] {0.5*x + 3};

			\draw (axis cs:3,4.5) -- (axis cs:5,4.5) -- (axis cs:5,5.5);
			\node at (axis cs:4,4.5) [below] {$1$};
			\node at (axis cs:5,5) [right] {$n$};

			\coordinate (title) at (axis cs:4.5,10) {};
			\coordinate (y intercept) at (axis cs:0,3) {};
		\end{axis}

		\node at (title) [above] {$y = a x^n$};
		\node at (y intercept) [above left] {$\log a$};
	\end{tikzpicture}
\end{minipage}\hfill
\begin{minipage}{0.49\linewidth}
	\begin{tikzpicture}
		\begin{axis}[
			grid=none,
			no marks,
			axis lines=middle,
			ticks=none,
			xmin=-1,
			xmax=10,
			ymin=-1,
			ymax=10,
			xlabel=$x$,
			ylabel=$\log y$,
			width=\linewidth
		]
			\addplot [thick, domain=-1:10] {0.5*x + 3};

			\draw (axis cs:3,4.5) -- (axis cs:5,4.5) -- (axis cs:5,5.5);
			\node at (axis cs:4,4.5) [below] {$1$};
			\node at (axis cs:5,5) [right] {$\log b$};

			\coordinate (title) at (axis cs:4.5,10) {};
			\coordinate (y intercept) at (axis cs:0,3) {};
		\end{axis}

		\node at (title) [above] {$y = a b^x$};
		\node at (y intercept) [above left] {$\log a$};
	\end{tikzpicture}
\end{minipage}
\end{figure}

\end{document}
