\documentclass[../main.tex]{subfile}

\begin{document}

\topictitle{Sequences and Series}

\sectitle{Arithmetic}

\vspace{-0.8em}
\begin{center}
	$a$ is the first term. $d$ is the common difference. Counting starts from $n = 1$.

	The $n$th term of the sequence is given by:
	\begin{empheq}[box=\rememberBox]{align*}
		u_n = a + (n - 1)d
	\end{empheq}

	The sum of the first $n$ terms (inclusive) of the series is given by:
	\begin{empheq}[box=\formulaBookBox]{align*}
		S_n = \frac{n}{2} (2a + (n - 1)d)
	\end{empheq}

	Or \formulaBookBox{$S_n = \frac{n}{2}(a + l)$} where $l$ is the last term.
\end{center}

\sectitle{Geometric}

\vspace{-0.8em}
\begin{center}
	$a$ is the first term. $r$ is the common ratio. Counting starts from $n = 1$.

	The $n$th term of the sequence is given by:
	\begin{empheq}[box=\rememberBox]{align*}
		u_n = ar^{n - 1}
	\end{empheq}

	The sum of the first $n$ terms (inclusive) of the series is given by:
	$$\formulaBookBox{$\displaystyle S_n = \frac{a(1 - r^n)}{1 - r} = \frac{a(r^n - 1)}{r - 1}$}
	\ \constraint{r \ne 1}$$

	The sum to infinity of a geometric series is \textbf{only valid} when $|r| < 1$ and is given by:
	$$\formulaBookBox{$\displaystyle S_\infty = \frac{a}{1 - r}$}
	\ \constraint{|r| < 1}$$
\end{center}

\end{document}
