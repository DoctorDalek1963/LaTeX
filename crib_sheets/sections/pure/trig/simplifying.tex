\documentclass[../main.tex]{subfile}

\begin{document}

\topictitle{Simplifying combinations of trig functions}

{\large
\begin{empheq}[box=\rememberBox]{gather*}
	a\sin x \pm b\cos x \equiv R\sin(x \pm \alpha)\\
	a\cos x \pm b\sin x \equiv R\cos(x \mp \alpha)\\
	R\cos\alpha = a \text{ and } R\sin\alpha = b \implies \tan\alpha = \frac{b}{a}\\
	R = \sqrt{a^2 + b^2}
\end{empheq}}

Simplify the expression to use the $R$ form with whichever trig function comes first in the pair. Then $\tan\alpha \equiv \dfrac{b}{a}$.

\end{document}
