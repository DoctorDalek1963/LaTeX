\documentclass[../main.tex]{subfile}

\begin{document}

\topictitle{Probability Distributions}

\sectitle{Binomial}

The binomial distribution is used to model a situation with a fixed number of independent trials each with a constant probability of success.

You can model $X$ as a binomial distribution if:
\begin{itemize}
	\item There a fixed number of trials, $n$
	\item Each trial must succeed or fail
	\item There is a fixed probability of success, $p$
	\item Each trial is independent
\end{itemize}

\begin{center}
	If $X \sim B(n, p)$, then $\boxed{P(X = x) = \binom{n}{x} p^x (1 - p)^{n - x}}\ \constraint{0 \leq x \leq n}$
\end{center}

\sectitle{Normal}

The normal distribution $X \sim N(\mu, \sigma^2)$ is symmetrical, meaning the mean and median are equal.

\begin{figure}[h]
\begin{center}
\begin{tikzpicture}
	\begin{axis}[
		default axis,
		axis lines*=middle,
		xmin=-5,
		xmax=5,
		ymin=0,
		ymax=0.42,
		width=0.8\linewidth,
		grid=none,
		xlabel={},
		ylabel={},
		xtick={-5,-4,-3,-2,-1,1,2,3,4,5},
		xticklabels={
			$-5\sigma$, $-4\sigma$, $-3\sigma$, $-2\sigma$, $-\sigma$,
			$\sigma$, $2\sigma$, $3\sigma$, $4\sigma$, $5\sigma$,
		},
		ytick style={draw=none},
		yticklabels={},
	]
		\addplot [thick] {(2*\npi)^(-1/2)*exp(-(x^2)/2};
	\end{axis}
\end{tikzpicture}
\end{center}
\end{figure}

\end{document}
