\documentclass[../main.tex]{subfile}

\begin{document}

\topictitle{Errors}

The first step of an errors problem is always to find the critical region. The critical region is the region where $\text{H}_0$ would be \textit{rejected}.

\begin{center}
\large
\begin{tabular}{|c|c|l|l|}
	\hline
	\phantom{} & \multicolumn{3}{c|}{Truth}\\
	\hline
	\multirow{3}{*}{\rotatebox[origin=c]{90}{Conc.}} & & $\text{H}_0$ & $\text{H}_1$ \\
	\cline{2-4}
	& $\text{H}_0$ & True negative & False negative (Type II error)\\
	\cline{2-4}
	& $\text{H}_1$ & False positive (Type I error) & True positive (Power)\\
	\hline
\end{tabular}
\end{center}

The \textbf{Type I error} or \textbf{size} is the probability of \textit{being in} the critical region with the \textit{original} parameter.

The \textbf{Type II error} is the probability of \textit{not being in} the critical region with a \textit{new} parameter.

The \textbf{Power} (true positive) is the probability of \textit{being in} the critical region with a \textit{new} parameter. The power function is a function to find the power given a specific new parameter.

\end{document}
