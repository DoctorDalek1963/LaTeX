\documentclass[../main.tex]{subfile}

\begin{document}

\topictitle{Correlation}

\begin{figure}[H]
	\begin{minipage}{0.45\linewidth}
		\begin{figure}[H]
			\centering
			\begin{tikzpicture}[
				scale=0.35,
				cross/.style={
					cross out, draw,
					minimum size=2*(#1-\pgflinewidth),
					inner sep=0pt, outer sep=0pt
				},
				cross/.default={4pt}
			]
				\draw (-1, 0) -- (20, 0);
				\draw (0, -1) -- (0, 20);

				\node[cross] at (4, 6.5) {};
				\node[cross] at (4, 6) {};
				\node[cross] at (8, 9.5) {};
				\node[cross] at (7, 11.5) {};
				\node[cross] at (12, 16.5) {};
				\node[cross] at (12, 18.5) {};
				\node[cross] at (3, 5) {};
				\node[cross] at (10, 11.5) {};

				\clip (1, 1) rectangle (19, 19);
				\draw (0, 1.847826087) -- (20, {25.86956522});
			\end{tikzpicture}
			\caption{A strong positive correlation}
		\end{figure}
	\end{minipage}\hfill
	\begin{minipage}{0.5\linewidth}
		The \textbf{product moment correlation coefficient} or \enquote*{Pearson's correlation coefficient} is a measure of correlation between two variables. It is often called $r$ and is measured in the range $[-1, 1]$. $r = \pm 1$ means the data perfectly follows a positive or negative correlation respectively.

		\vspace{5ex}
		\sectitle{Hypothesis testing}

		Use $H_0: \rho = 0$ and $H_1: \rho \lessgtr 0$ or $H_1: \rho \neq 0$. Get $r$ from your calculator (use section 6 and option 4 regression calc) and $\rho$ from the table in the formula book by finding the sample size and significance level. Remember to halve the significance level if the test is two-tailed.
	\end{minipage}
\end{figure}

\end{document}
