\documentclass[../main.tex]{subfile}

\begin{document}

\topictitle{Probability Distributions}

\sectitle{Binomial}

The binomial distribution is used to model a situation with a fixed number of independent trials each with a constant probability of success.

You can model $X$ as a binomial distribution if:
\begin{itemize}
	\item There a fixed number of trials, $n$
	\item Each trial must succeed or fail
	\item There is a fixed probability of success, $p$
	\item Each trial is independent
\end{itemize}

\begin{center}
	If $X \sim \text{B}(n, p)$, then \formulaBookBox{$\displaystyle P(X = x) = \binom{n}{x} p^x (1 - p)^{n - x}$}
	\ $\constraint{0 \leq x \leq n}$
\end{center}

\sectitle{Normal}

The normal distribution $X \sim \text{N}(\mu, \sigma^2)$ is symmetrical, meaning the mean and median are equal.

\begin{figure}[h]
\begin{center}
\begin{minipage}{0.5\linewidth}
	\setlength{\parskip}{1ex}
	When doing questions that involve the normal distribution, sketching the bell curve on the right is always a good idea.\\

	The standard normal distribution $Z \sim \text{N}(0, 1^2)$ is very useful, since it allows you to find values for $\mu$ and $\sigma$ when they're unknown. The normal random variable $X$ can be coded using \rememberBox{$\displaystyle Z = \frac{X - \mu}{\sigma}$} and you can use this equation to find the unknown parameters of $X$.\\

	The \enquote{Normal CD} function on your calculator will calculate the area between an upper and lower bound on the bell curve. The \enquote{Inverse Normal} function will find a value for which the area to the \underline{\textit{left}} of that value is the area you specify.
\end{minipage}\hfill
\begin{minipage}{0.48\linewidth}
	\begin{tikzpicture}
		\begin{axis}[
			grid=both,
			samples=1000,
			no marks,
			axis lines=middle,
			xlabel=$x$,
			ylabel=$y$,
			axis lines*=middle,
			xmin=-5,
			xmax=5,
			ymin=0,
			ymax=0.42,
			width=\linewidth,
			height=0.65\linewidth,
			grid=none,
			xlabel={},
			ylabel={},
			xtick={-5,-4,-3,-2,-1,1,2,3,4,5},
			xticklabels={
				$-5\sigma$, $-4\sigma$, $-3\sigma$, $-2\sigma$, $-\sigma$,
				$\sigma$, $2\sigma$, $3\sigma$, $4\sigma$, $5\sigma$,
			},
			ytick style={draw=none},
			yticklabels={},
		]
			\addplot [thick] {(2*\npi)^(-1/2)*exp(-(x^2)/2};
		\end{axis}
	\end{tikzpicture}
\end{minipage}
\end{center}
\end{figure}

\end{document}
