\documentclass[../main.tex]{subfile}

\begin{document}

\topictitle{Hypothesis Testing}

Every hypothesis test has two hypotheses:
\begin{align*}
	H_0 & \text{ : The null hypothesis - this is what you assume to be true by default}\\
	H_1 & \text{ : The alternative hypothesis}
\end{align*}

The hypotheses are written in different forms depending on whether the test is one- or two-tailed.

\begin{minipage}{0.49\linewidth}
	\begin{center}
		\underline{One-tailed:}
		\begin{align*}
			H_0:&\ p = k\\
			H_1:&\ p \lessgtr k
		\end{align*}
	\end{center}
\end{minipage}\hfill
\begin{minipage}{0.49\linewidth}
	\begin{center}
		\underline{Two-tailed:}
		\begin{align*}
			H_0:&\ p = k\\
			H_1:&\ p \ne k
		\end{align*}
	\end{center}
\end{minipage}

If the question says that someone measured and got a value, then you plug that value into a probability calculation with the parameters from the null hypothesis, with the inequality sign in the same direction as the alternative hypothesis. If the probability of the event happening when assuming $H_0$ is less than the level of significance, then we reject $H_0$ and accept the alternative hypothesis.

A critical value is the smallest or largest value (depending on the direction of the inequality) obtained by a random variable such that $H_0$ would be rejected. Finding a critical value is often best done with the tables in the back of the book.

\end{document}
