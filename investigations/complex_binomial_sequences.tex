\documentclass[a4paper]{article}
\usepackage[utf8]{inputenc}
\usepackage{amsmath, amssymb, amsthm}
\usepackage{gensymb}
\usepackage[nodayofweek]{datetime}

\usepackage{tikz}

% Set size of text area with total parameter
\usepackage[a4paper, total={135mm, 255mm}]{geometry}

\title{Complex Binomial Sequences}
\author{Dyson}
\date{\today}

\newcommand{\abs}[1]{\left| #1 \right|}
\newcommand{\opi}{(1 + i)}
\newcommand{\mopi}{(-1 + i)}

\begin{document}

\maketitle

% Set paragraph spacing here to avoid messing with title
\setlength{\parindent}{0em}
\setlength{\parskip}{1em}

Throughout this paper, all $n \in \mathbb{N}$ and $0 \in \mathbb{N}$.

\section{Looking at $\opi^n$}

The sequence $\opi^n$ creates an interesting graph when plotted sequentially on an Argand diagram.

\begin{center}
\resizebox{135mm}{!}{
\begin{tikzpicture}
    \begin{scope}[thick,font=\scriptsize]
        % Axes
        \draw (-18,0) -- (18,0);
        \draw (0,-18) -- (0,18);

        % Axes labels
        \foreach \n in {-17,...,-1,1,2,...,17}{
            \draw (\n,-3pt) -- (\n,3pt);
            \draw (-3pt,\n) -- (3pt,\n);
        }

        % Draw the actual plot of the sequence
        \draw (1,0) -- (1,1) -- (0,2) -- (-2,2) -- (-4,0) -- (-4,-4) -- (0,-8) -- (8,-8) -- (16,0) -- (16,16);
    \end{scope}
\end{tikzpicture}
}
\end{center}

Initially, I confused this shape for a coarse approximation of a Golden Spiral. If this were the case, then the ratio between the moduli of successive elements of the sequence would tend to $\phi$ as $n$ grows to $\infty$. Here's an example:

\begin{align*}
\dfrac{\abs{\opi^3}}{\abs{\opi^2}} = \dfrac{\abs{-2 + 2i}}{\abs{2i}} = \dfrac{\sqrt{8}}{2} = \dfrac{2\sqrt{2}}{2} = \sqrt{2}
\end{align*}

The ratio is $\sqrt{2}$, which is not $\phi$. This ratio is true for all $\dfrac{\abs{\opi^{n + 1}}}{\abs{\opi^n}}$, meaning this shape is not a Golden Spiral.

\begin{proof}
Let $z_1$ and $z_2$ be two consecutive elements of the sequence $\opi^n$.

We want the ratio between the moduli of these elements, $\dfrac{\abs{z_2}}{\abs{z_1}}$.

Let $f(n) = \dfrac{\abs{\opi^{n + 1}}}{\abs{\opi^n}}$.

We can easily show that $f(0) = \dfrac{\abs{\opi^1}}{\abs{\opi^0}} = \dfrac{\abs{1 + i}}{\abs{1}} = \sqrt{2}$.

We can then show:
\begin{gather*}
f(n) = \dfrac{\abs{\opi^{n + 1}}}{\abs{\opi^n}}\\[0.5em]
= \dfrac{\abs{\opi\opi^n}}{\abs{\opi\opi^{n - 1}}} = \dfrac{\abs{\opi}}{\abs{\opi}} \times \dfrac{\abs{\opi^n}}{\abs{\opi^{n - 1}}}\\[0.5em]
= 1 \times \dfrac{\abs{\opi^n}}{\abs{\opi^{n - 1}}} = f(n - 1)
\end{gather*}

We can do this because for $z, w \in \mathbb{C}$, $\abs{zw} = \abs{z} \times \abs{w}$.

We've now shown that for $n > 0$, $f(n) = f(n - 1)$. This always recurs to the base case of $f(0)$ and means that all $f(n) = \sqrt{2}$.

This means that the ratio of the moduli between any two consecutive members of the sequence $\opi^n$ is always $\sqrt{2}$.
\end{proof}

\newpage

\section{Looking at $\mopi^n$}

If we look at the series $(1 - i)^n$ and plot it sequentially on an Argand diagram, then we get the same spiral as when plotting $\opi$ but reflected in the real axis.

However, if we look at $\mopi^n$, then we get something completely different.

\begin{center}
    \resizebox{135mm}{!}{
        \begin{tikzpicture}
        \begin{scope}[thick,font=\scriptsize]
        % Axes
        \draw (-18,0) -- (18,0);
        \draw (0,-18) -- (0,18);

        % Axes labels
        \foreach \n in {-17,...,-1,1,2,...,17}{
            \draw (\n,-3pt) -- (\n,3pt);
            \draw (-3pt,\n) -- (3pt,\n);
        }

        % Draw the actual plot of the sequence
        \draw (1,0) -- (-1,1) -- (0,-2) -- (2,2) -- (-4,0) -- (4,-4) -- (0,8) -- (-8,-8) -- (16,0) -- (-16,16);
        \end{scope}
        \end{tikzpicture}
    }
\end{center}

This is a very interesting plot. We can use a very similar proof to the one at the end of \S1 to prove that the ratio between the moduli of any two elements of the sequence is $\sqrt{2}$. However, the more interesting thing with this sequence is not the moduli of the elements, but the arguments. How does the angle of each point relative to the origin change?

Looking at the sequence $\arg(n)$ (in degrees) for $n \in \{0,1,...,10\}$, we get
% This text environment is to add spaces between each element. Yes, a text env in a math env is a bodge. Sue me.
$$\text{0, 135, 270, 405, 540, 675, 810, 945, 1080, 1215, 1350}$$

If we take each of these elements mod 360, we get
$$\text{0, 135, 270, 45, 180, 315, 90, 225, 0, 135, 270}$$

The arguments of the members of $\mopi^n$ rotate around the unit circle with a period of 8. This is the same period as $\opi^n$.

But, despite the two sequences having the same ratio of moduli, and the same rotation period, this second sequence creates a completely different graph. This is because the rotation of each term of $\opi^n$ relative to the last is 45. This means that it takes a whole rotation of 8 terms to get back to the initial angle.

However, the relative rotation of $\mopi^n$ is 135, meaning that we after 3 terms, we've rotated $45\degree$ from the initial value. This creates a very interesting pattern of triangles on the plot.

\end{document}
